\documentclass{article}

%% Created with wxMaxima 16.04.2

\setlength{\parskip}{\medskipamount}
\setlength{\parindent}{0pt}
\usepackage[utf8]{inputenc}
\DeclareUnicodeCharacter{00B5}{\ensuremath{\mu}}
\usepackage{graphicx}
\usepackage{color}
\usepackage{amsmath}
\usepackage{ifthen}
\newsavebox{\picturebox}
\newlength{\pictureboxwidth}
\newlength{\pictureboxheight}
\newcommand{\includeimage}[1]{
    \savebox{\picturebox}{\includegraphics{#1}}
    \settoheight{\pictureboxheight}{\usebox{\picturebox}}
    \settowidth{\pictureboxwidth}{\usebox{\picturebox}}
    \ifthenelse{\lengthtest{\pictureboxwidth > .95\linewidth}}
    {
        \includegraphics[width=.95\linewidth,height=.80\textheight,keepaspectratio]{#1}
    }
    {
        \ifthenelse{\lengthtest{\pictureboxheight>.80\textheight}}
        {
            \includegraphics[width=.95\linewidth,height=.80\textheight,keepaspectratio]{#1}
            
        }
        {
            \includegraphics{#1}
        }
    }
}
\newlength{\thislabelwidth}
\DeclareMathOperator{\abs}{abs}
\usepackage{animate} % This package is required because the wxMaxima configuration option
                      % "Export animations to TeX" was enabled when this file was generated.

\definecolor{labelcolor}{RGB}{100,0,0}

\usepackage{fullpage}
\usepackage{amssymb}
\usepackage{enumerate}
\usepackage[bookmarks=false,pdfstartview={FitH},colorlinks=true,urlcolor=blue]{hyperref}
\usepackage{bookmark}
\usepackage{mathtools}

\begin{document}

\pagebreak{}
{\Huge {\sc One flux example two ways}}
\setcounter{section}{0}
\setcounter{subsection}{0}
\setcounter{figure}{0}


\hypersetup{pdfauthor={Daniel Volinski},
            pdftitle={Flux example},
            pdfsubject={Multivariable Calculus},
            pdfkeywords={Dr. Bevin Maultsby}}

Based on Dr. Bevin Maultsby Playlist
\href{https://www.youtube.com/watch?v=joy7ntJmh_0&list=PLBEl4BT8wUgOqJCBijQxClMIuumgtMinc&index=107}
{One flux example two ways: using Stokes' and the Divergence Theorem}

Written by Daniel Volinski at \href{mailto:danielvolinski@yahoo.es}{danielvolinski@yahoo.es}



\noindent
%%%%%%%%%%%%%%%
%%% INPUT:
\begin{minipage}[t]{8ex}\color{red}\bf
(\%{}i2) 
\end{minipage}
\begin{minipage}[t]{\textwidth}\color{blue}\tt
info:build\_info()\$info\ensuremath{@}version;
\end{minipage}
%%% OUTPUT:
\[\displaystyle
\tag{\%{}o2}\label{o2} 
\mbox{}
\]5.38.1



\noindent
%%%%%%%%%%%%%%%
%%% INPUT:
\begin{minipage}[t]{8ex}\color{red}\bf
(\%{}i2) 
\end{minipage}
\begin{minipage}[t]{\textwidth}\color{blue}\tt
reset()\$kill(all)\$
\end{minipage}


\noindent
%%%%%%%%%%%%%%%
%%% INPUT:
\begin{minipage}[t]{8ex}\color{red}\bf
(\%{}i1) 
\end{minipage}
\begin{minipage}[t]{\textwidth}\color{blue}\tt
derivabbrev:true\$
\end{minipage}


\noindent
%%%%%%%%%%%%%%%
%%% INPUT:
\begin{minipage}[t]{8ex}\color{red}\bf
(\%{}i2) 
\end{minipage}
\begin{minipage}[t]{\textwidth}\color{blue}\tt
ratprint:false\$
\end{minipage}


\noindent
%%%%%%%%%%%%%%%
%%% INPUT:
\begin{minipage}[t]{8ex}\color{red}\bf
(\%{}i3) 
\end{minipage}
\begin{minipage}[t]{\textwidth}\color{blue}\tt
fpprintprec:5\$
\end{minipage}


\noindent
%%%%%%%%%%%%%%%
%%% INPUT:
\begin{minipage}[t]{8ex}\color{red}\bf
(\%{}i4) 
\end{minipage}
\begin{minipage}[t]{\textwidth}\color{blue}\tt
load(linearalgebra)\$
\end{minipage}


\noindent
%%%%%%%%%%%%%%%
%%% INPUT:
\begin{minipage}[t]{8ex}\color{red}\bf
(\%{}i5) 
\end{minipage}
\begin{minipage}[t]{\textwidth}\color{blue}\tt
if get('draw,'version)=false then load(draw)\$
\end{minipage}
%%% OUTPUT:
%%%%%%%%%%%%%%%


\noindent
%%%%%%%%%%%%%%%
%%% INPUT:
\begin{minipage}[t]{8ex}\color{red}\bf
(\%{}i6) 
\end{minipage}
\begin{minipage}[t]{\textwidth}\color{blue}\tt
wxplot\_size:[1024,768]\$
\end{minipage}


\noindent
%%%%%%%%%%%%%%%
%%% INPUT:
\begin{minipage}[t]{8ex}\color{red}\bf
(\%{}i7) 
\end{minipage}
\begin{minipage}[t]{\textwidth}\color{blue}\tt
set\_draw\_defaults(xtics=1,ytics=1,ztics=1,xyplane=0,nticks=100,\\
                  xaxis=true,xaxis\_type=dots,xaxis\_width=3,\\
                  yaxis=true,yaxis\_type=dots,yaxis\_width=3,\\
                  zaxis=true,zaxis\_type=dots,zaxis\_width=3,\\
                  background\_color=light\_gray)\$
\end{minipage}


\noindent
%%%%%%%%%%%%%%%
%%% INPUT:
\begin{minipage}[t]{8ex}\color{red}\bf
(\%{}i8) 
\end{minipage}
\begin{minipage}[t]{\textwidth}\color{blue}\tt
if get('vect,'version)=false then load(vect)\$
\end{minipage}


\noindent
%%%%%%%%%%%%%%%
%%% INPUT:
\begin{minipage}[t]{8ex}\color{red}\bf
(\%{}i9) 
\end{minipage}
\begin{minipage}[t]{\textwidth}\color{blue}\tt
norm(u):=block(ratsimp(radcan(\ensuremath{\sqrt{}}(u.u))))\$
\end{minipage}


\noindent
%%%%%%%%%%%%%%%
%%% INPUT:
\begin{minipage}[t]{8ex}\color{red}\bf
(\%{}i10) 
\end{minipage}
\begin{minipage}[t]{\textwidth}\color{blue}\tt
normalize(v):=block(v/norm(v))\$
\end{minipage}


\noindent
%%%%%%%%%%%%%%%
%%% INPUT:
\begin{minipage}[t]{8ex}\color{red}\bf
(\%{}i11) 
\end{minipage}
\begin{minipage}[t]{\textwidth}\color{blue}\tt
angle(u,v):=block([junk:radcan(\ensuremath{\sqrt{}}((u.u)*(v.v)))],acos(u.v/junk))\$
\end{minipage}


\noindent
%%%%%%%%%%%%%%%
%%% INPUT:
\begin{minipage}[t]{8ex}\color{red}\bf
(\%{}i12) 
\end{minipage}
\begin{minipage}[t]{\textwidth}\color{blue}\tt
mycross(va,vb):=[va[2]*vb[3]-va[3]*vb[2],va[3]*vb[1]-va[1]*vb[3],va[1]*vb[2]-va[2]*vb[1]]\$
\end{minipage}


\noindent
%%%%%%%%%%%%%%%
%%% INPUT:
\begin{minipage}[t]{8ex}\color{red}\bf
(\%{}i13) 
\end{minipage}
\begin{minipage}[t]{\textwidth}\color{blue}\tt
if get('cartan,'version)=false then load(cartan)\$
\end{minipage}


\noindent
%%%%%%%%%%%%%%%
%%% INPUT:
\begin{minipage}[t]{8ex}\color{red}\bf
(\%{}i14) 
\end{minipage}
\begin{minipage}[t]{\textwidth}\color{blue}\tt
declare(trigsimp,evfun)\$
\end{minipage}
\pagebreak


Let $S$ be the surface $z=x^2+y^2$, $0\leq z\leq 16$, oriented
with outward-pointing normal vectors.

Let $\vec{F}(x,y,z)=\langle{2\,y,-2\,x,-8\,x^2+12\,y+cos(z^2)}\rangle$.

Compute $\iint_S\nabla\times\vec{F}\cdot\mathrm{d}\vec{S}$ using ...
\begin{enumerate}[(a)]
\item Stokes' theorem
\item the Divergence theorem
\end{enumerate}


\textbf{Stokes' theorem}
$$\iint_S\nabla\times\vec{F}\cdot\mathrm{d}\vec{S}=
\oint_{\partial S}\vec{F}\cdot\mathrm{d}\vec{r}$$


\textbf{Define the space} $\mathbb{R}^3$



\noindent
%%%%%%%%%%%%%%%
%%% INPUT:
\begin{minipage}[t]{8ex}\color{red}\bf
(\%{}i15) 
\end{minipage}
\begin{minipage}[t]{\textwidth}\color{blue}\tt
\ensuremath{\zeta}:[x,y,z]\$
\end{minipage}


\noindent
%%%%%%%%%%%%%%%
%%% INPUT:
\begin{minipage}[t]{8ex}\color{red}\bf
(\%{}i16) 
\end{minipage}
\begin{minipage}[t]{\textwidth}\color{blue}\tt
dim:length(\ensuremath{\zeta})\$
\end{minipage}


\noindent
%%%%%%%%%%%%%%%
%%% INPUT:
\begin{minipage}[t]{8ex}\color{red}\bf
(\%{}i17) 
\end{minipage}
\begin{minipage}[t]{\textwidth}\color{blue}\tt
scalefactors(\ensuremath{\zeta})\$
\end{minipage}


\noindent
%%%%%%%%%%%%%%%
%%% INPUT:
\begin{minipage}[t]{8ex}\color{red}\bf
(\%{}i18) 
\end{minipage}
\begin{minipage}[t]{\textwidth}\color{blue}\tt
init\_cartan(\ensuremath{\zeta})\$
\end{minipage}

\textbf{Vector field} $\vec{F}\in\mathbb{R}^3$



\noindent
%%%%%%%%%%%%%%%
%%% INPUT:
\begin{minipage}[t]{8ex}\color{red}\bf
(\%{}i19) 
\end{minipage}
\begin{minipage}[t]{\textwidth}\color{blue}\tt
ldisplay(F:[2*y,-2*x,-8*x\ensuremath{^2}+12*y+cos(z\ensuremath{^2})])\$
\end{minipage}
%%% OUTPUT:
\[\displaystyle
\tag{\%{}t19}\label{t19} 
F=[2y,-2x,\cos{\left( {{z}^{2}}\right) }+12y-8{{x}^{2}}]\mbox{}
\]
%%%%%%%%%%%%%%%

\textbf{3D Direction field}



\noindent
%%%%%%%%%%%%%%%
%%% INPUT:
\begin{minipage}[t]{8ex}\color{red}\bf
(\%{}i21) 
\end{minipage}
\begin{minipage}[t]{\textwidth}\color{blue}\tt
/* vector origins are {(x,y,z)| x,y=1,...,5}  */\\
coord:setify(makelist(k,k,-4,4))\$\\
points3d:listify(cartesian\_product(coord,coord,coord))\$
\end{minipage}


\noindent
%%%%%%%%%%%%%%%
%%% INPUT:
\begin{minipage}[t]{8ex}\color{red}\bf
(\%{}i23) 
\end{minipage}
\begin{minipage}[t]{\textwidth}\color{blue}\tt
/* compute vectors at the given points  */\\
define(vf3d(x,y,z),vector(\ensuremath{\zeta},F))\$\\
vect3:makelist(vf3d(k[1],k[2],k[3]),k,points3d)\$
\end{minipage}


\noindent
%%%%%%%%%%%%%%%
%%% INPUT:
\begin{minipage}[t]{8ex}\color{red}\bf
(\%{}i24) 
\end{minipage}
\begin{minipage}[t]{\textwidth}\color{blue}\tt
wxdraw3d([head\_length=0.1,color=blue,head\_angle=25,unit\_vectors=true],vect3)\$
\end{minipage}
%%% OUTPUT:
\[\displaystyle
\tag{\%{}t24}\label{t24} 
\includegraphics[width=.95\linewidth,height=.80\textheight,keepaspectratio]{One flux example two ways_img/One flux example two ways_1}\mbox{}
\]
%%%%%%%%%%%%%%%
\pagebreak


\textbf{Calculate} $\nabla\times\vec{F}\in\mathbb{R}^3$



\noindent
%%%%%%%%%%%%%%%
%%% INPUT:
\begin{minipage}[t]{8ex}\color{red}\bf
(\%{}i25) 
\end{minipage}
\begin{minipage}[t]{\textwidth}\color{blue}\tt
ldisplay(curlF:ev(express(curl(F)),diff))\$
\end{minipage}
%%% OUTPUT:
\[\displaystyle
\tag{\%{}t25}\label{t25} 
\mathit{curlF}=[12,16x,-4]\mbox{}
\]
%%%%%%%%%%%%%%%

\textbf{Work form} $\alpha=F^\flat\in\mathcal{A}^1(\mathbb{R}^3)$



\noindent
%%%%%%%%%%%%%%%
%%% INPUT:
\begin{minipage}[t]{8ex}\color{red}\bf
(\%{}i26) 
\end{minipage}
\begin{minipage}[t]{\textwidth}\color{blue}\tt
ldisplay(\ensuremath{\alpha}:F.cartan\_basis)\$
\end{minipage}
%%% OUTPUT:
\[\displaystyle
\tag{\%{}t26}\label{t26} 
\mathit{\ensuremath{\alpha}}=\left( \cos{\left( {{z}^{2}}\right) }+12y-8{{x}^{2}}\right) \,\mathit{dz}-2x\,\mathit{dy}+2y\,\mathit{dx}\mbox{}
\]
%%%%%%%%%%%%%%%

\textbf{Calculate} $\mathrm{d}\alpha\in\mathcal{A}^2(\mathbb{R}^3)$



\noindent
%%%%%%%%%%%%%%%
%%% INPUT:
\begin{minipage}[t]{8ex}\color{red}\bf
(\%{}i27) 
\end{minipage}
\begin{minipage}[t]{\textwidth}\color{blue}\tt
ldisplay(d\ensuremath{\alpha}:edit(ext\_diff(\ensuremath{\alpha})))\$
\end{minipage}
%%% OUTPUT:
\[\displaystyle
\tag{\%{}t27}\label{t27} 
\mathit{d\ensuremath{\alpha}}=12\mathit{dy}\,\mathit{dz}-16x\,\mathit{dx}\,\mathit{dz}-4\mathit{dx}\,\mathit{dy}\mbox{}
\]
%%%%%%%%%%%%%%%

\textbf{Calculate} $\nabla\cdot\vec{F}\in\mathbb{R}$



\noindent
%%%%%%%%%%%%%%%
%%% INPUT:
\begin{minipage}[t]{8ex}\color{red}\bf
(\%{}i28) 
\end{minipage}
\begin{minipage}[t]{\textwidth}\color{blue}\tt
ldisplay(divF:ev(express(div(F)),diff))\$
\end{minipage}
%%% OUTPUT:
\[\displaystyle
\tag{\%{}t28}\label{t28} 
\mathit{divF}=-2z\,\sin{\left( {{z}^{2}}\right) }\mbox{}
\]
%%%%%%%%%%%%%%%

\textbf{Flux form} $\beta\in\mathcal{A}^2(\mathbb{R}^3)$



\noindent
%%%%%%%%%%%%%%%
%%% INPUT:
\begin{minipage}[t]{8ex}\color{red}\bf
(\%{}i29) 
\end{minipage}
\begin{minipage}[t]{\textwidth}\color{blue}\tt
ldisplay(\ensuremath{\beta}:F[1]*cartan\_basis[2]\ensuremath{\sim }cartan\_basis[3]+\\
           F[2]*cartan\_basis[3]\ensuremath{\sim }cartan\_basis[1]+\\
           F[3]*cartan\_basis[1]\ensuremath{\sim }cartan\_basis[2])\$
\end{minipage}
%%% OUTPUT:
\[\displaystyle
\tag{\%{}t29}\label{t29} 
\mathit{\ensuremath{\beta}}=2y\,\mathit{dy}\,\mathit{dz}+2x\,\mathit{dx}\,\mathit{dz}+\left( \cos{\left( {{z}^{2}}\right) }+12y-8{{x}^{2}}\right) \,\mathit{dx}\,\mathit{dy}\mbox{}
\]
%%%%%%%%%%%%%%%

\textbf{Calculate} $\mathrm{d}\beta\in\mathcal{A}^3(\mathbb{R}^3)$



\noindent
%%%%%%%%%%%%%%%
%%% INPUT:
\begin{minipage}[t]{8ex}\color{red}\bf
(\%{}i30) 
\end{minipage}
\begin{minipage}[t]{\textwidth}\color{blue}\tt
ldisplay(d\ensuremath{\beta}:edit(ext\_diff(\ensuremath{\beta})))\$
\end{minipage}
%%% OUTPUT:
\[\displaystyle
\tag{\%{}t30}\label{t30} 
\mathit{d\ensuremath{\beta}}=-2z\,\sin{\left( {{z}^{2}}\right) }\,\mathit{dx}\,\mathit{dy}\,\mathit{dz}\mbox{}
\]
%%%%%%%%%%%%%%%
\pagebreak


\textbf{Draw the paraboloid and its boundary}



\noindent
%%%%%%%%%%%%%%%
%%% INPUT:
\begin{minipage}[t]{8ex}\color{red}\bf
(\%{}i31) 
\end{minipage}
\begin{minipage}[t]{\textwidth}\color{blue}\tt
wxdraw3d(axis\_3d=false,xyplane=0,nticks=100,\\
         x\_voxel=20,y\_voxel=20,z\_voxel=20,\\
         implicit(z=x\ensuremath{^2}+y\ensuremath{^2},x,-4,4,y,-4,4,z,0,16),\\
         line\_width=3,color=black,\\
         parametric(4*cos(-t),4*sin(-t),16,t,0,2*\ensuremath{\pi}))\$
\end{minipage}
%%% OUTPUT:
\[\displaystyle
\tag{\%{}t31}\label{t31} 
\includegraphics[width=.95\linewidth,height=.80\textheight,keepaspectratio]{One flux example two ways_img/One flux example two ways_2}\mbox{}
\]
%%%%%%%%%%%%%%%
\pagebreak


\textbf{Curve} $\vec{r}\in\mathbb{R}^3$



\noindent
%%%%%%%%%%%%%%%
%%% INPUT:
\begin{minipage}[t]{8ex}\color{red}\bf
(\%{}i32) 
\end{minipage}
\begin{minipage}[t]{\textwidth}\color{blue}\tt
ldisplay(r:[4*cos(-t),4*sin(-t),16])\$
\end{minipage}
%%% OUTPUT:
\[\displaystyle
\tag{\%{}t32}\label{t32} 
r=[4\cos{(t)},-4\sin{(t)},16]\mbox{}
\]
%%%%%%%%%%%%%%%

\textbf{Derivative of the curve} $\vec{r}$



\noindent
%%%%%%%%%%%%%%%
%%% INPUT:
\begin{minipage}[t]{8ex}\color{red}\bf
(\%{}i33) 
\end{minipage}
\begin{minipage}[t]{\textwidth}\color{blue}\tt
ldisplay(r\ensuremath{\backslash}':diff(r,t))\$
\end{minipage}
%%% OUTPUT:
\[\displaystyle
\tag{\%{}t33}\label{t33} 
\mathit{r'}=[-4\sin{(t)},-4\cos{(t)},0]\mbox{}
\]
%%%%%%%%%%%%%%%

\textbf{Calculate} $\vec{F}\circ\vec{r}$



\noindent
%%%%%%%%%%%%%%%
%%% INPUT:
\begin{minipage}[t]{8ex}\color{red}\bf
(\%{}i34) 
\end{minipage}
\begin{minipage}[t]{\textwidth}\color{blue}\tt
ldisplay(For:subst(map("=",\ensuremath{\zeta},r),F))\$
\end{minipage}
%%% OUTPUT:
\[\displaystyle
\tag{\%{}t34}\label{t34} 
\mathit{For}=[-8\sin{(t)},-8\cos{(t)},-48\sin{(t)}-128{{\cos{(t)}}^{2}}+\cos{(256)}]\mbox{}
\]
%%%%%%%%%%%%%%%

\textbf{Calculate} $\vec{r}^\ast\alpha\in\mathcal{A}^1(\mathbb{R}^3)$



\noindent
%%%%%%%%%%%%%%%
%%% INPUT:
\begin{minipage}[t]{8ex}\color{red}\bf
(\%{}i35) 
\end{minipage}
\begin{minipage}[t]{\textwidth}\color{blue}\tt
integrand:trigsimp(r\ensuremath{\backslash}'|subst(map("=",\ensuremath{\zeta},r),\ensuremath{\alpha}));
\end{minipage}
%%% OUTPUT:
\[\displaystyle
\tag{integrand}\label{integrand}
32\mbox{}
\]
%%%%%%%%%%%%%%%

\textbf{Calculate} $\vec{F}\cdot\vec{r}^{\prime}\in\mathbb{R}$



\noindent
%%%%%%%%%%%%%%%
%%% INPUT:
\begin{minipage}[t]{8ex}\color{red}\bf
(\%{}i36) 
\end{minipage}
\begin{minipage}[t]{\textwidth}\color{blue}\tt
integrand:trigsimp(For.r\ensuremath{\backslash}');
\end{minipage}
%%% OUTPUT:
\[\displaystyle
\tag{integrand}\label{integrand}
32\mbox{}
\]
%%%%%%%%%%%%%%%

\textbf{Flux integral}



\noindent
%%%%%%%%%%%%%%%
%%% INPUT:
\begin{minipage}[t]{8ex}\color{red}\bf
(\%{}i37) 
\end{minipage}
\begin{minipage}[t]{\textwidth}\color{blue}\tt
integrate(integrand,t,0,2*\ensuremath{\pi});
\end{minipage}
%%% OUTPUT:
\[\displaystyle
\tag{\%{}o37}\label{o37} 
64\ensuremath{\pi} \mbox{}
\]
%%%%%%%%%%%%%%%
\pagebreak


\textbf{the Divergence theorem}
$$\iint_S\nabla\times\vec{F}\cdot\mathrm{d}\vec{S}+
\iint_L\nabla\times\vec{F}\cdot\mathrm{d}\vec{S}=
\iiint_E\nabla\cdot(\nabla\times\vec{F})\mathrm{d}V=0$$


\textbf{Work form} $\gamma=(\nabla\times F)^\flat\in\mathcal{A}^1(\mathbb{R}^3)$



\noindent
%%%%%%%%%%%%%%%
%%% INPUT:
\begin{minipage}[t]{8ex}\color{red}\bf
(\%{}i38) 
\end{minipage}
\begin{minipage}[t]{\textwidth}\color{blue}\tt
ldisplay(\ensuremath{\gamma}:curlF.cartan\_basis)\$
\end{minipage}
%%% OUTPUT:
\[\displaystyle
\tag{\%{}t38}\label{t38} 
\mathit{\ensuremath{\gamma}}=-4\mathit{dz}+16x\,\mathit{dy}+12\mathit{dx}\mbox{}
\]
%%%%%%%%%%%%%%%

\textbf{Surface} $\vec{r}\in\mathbb{R}^3$



\noindent
%%%%%%%%%%%%%%%
%%% INPUT:
\begin{minipage}[t]{8ex}\color{red}\bf
(\%{}i39) 
\end{minipage}
\begin{minipage}[t]{\textwidth}\color{blue}\tt
ldisplay(r:[u*cos(v),u*sin(v),16])\$
\end{minipage}
%%% OUTPUT:
\[\displaystyle
\tag{\%{}t39}\label{t39} 
r=[u\,\cos{(v)},u\,\sin{(v)},16]\mbox{}
\]
%%%%%%%%%%%%%%%


\noindent
%%%%%%%%%%%%%%%
%%% INPUT:
\begin{minipage}[t]{8ex}\color{red}\bf
(\%{}i40) 
\end{minipage}
\begin{minipage}[t]{\textwidth}\color{blue}\tt
ldisplay(r\_u:diff(r,u))\$
\end{minipage}
%%% OUTPUT:
\[\displaystyle
\tag{\%{}t40}\label{t40} 
{{r}_{u}}=[\cos{(v)},\sin{(v)},0]\mbox{}
\]
%%%%%%%%%%%%%%%


\noindent
%%%%%%%%%%%%%%%
%%% INPUT:
\begin{minipage}[t]{8ex}\color{red}\bf
(\%{}i41) 
\end{minipage}
\begin{minipage}[t]{\textwidth}\color{blue}\tt
ldisplay(r\_v:diff(r,v))\$
\end{minipage}
%%% OUTPUT:
\[\displaystyle
\tag{\%{}t41}\label{t41} 
{{r}_{v}}=[-u\,\sin{(v)},u\,\cos{(v)},0]\mbox{}
\]
%%%%%%%%%%%%%%%

\textbf{Normal}



\noindent
%%%%%%%%%%%%%%%
%%% INPUT:
\begin{minipage}[t]{8ex}\color{red}\bf
(\%{}i42) 
\end{minipage}
\begin{minipage}[t]{\textwidth}\color{blue}\tt
ldisplay(n:trigsimp(mycross(r\_u,r\_v)))\$
\end{minipage}
%%% OUTPUT:
\[\displaystyle
\tag{\%{}t42}\label{t42} 
n=[0,0,u]\mbox{}
\]
%%%%%%%%%%%%%%%

\textbf{Integrand}



\noindent
%%%%%%%%%%%%%%%
%%% INPUT:
\begin{minipage}[t]{8ex}\color{red}\bf
(\%{}i43) 
\end{minipage}
\begin{minipage}[t]{\textwidth}\color{blue}\tt
integrand:n|\ensuremath{\gamma};
\end{minipage}
%%% OUTPUT:
\[\displaystyle
\tag{integrand}\label{integrand}
-4u\mbox{}
\]
%%%%%%%%%%%%%%%


\noindent
%%%%%%%%%%%%%%%
%%% INPUT:
\begin{minipage}[t]{8ex}\color{red}\bf
(\%{}i44) 
\end{minipage}
\begin{minipage}[t]{\textwidth}\color{blue}\tt
integrand:curlF.n;
\end{minipage}
%%% OUTPUT:
\[\displaystyle
\tag{integrand}\label{integrand}
-4u\mbox{}
\]
%%%%%%%%%%%%%%%

\textbf{Flux integral}



\noindent
%%%%%%%%%%%%%%%
%%% INPUT:
\begin{minipage}[t]{8ex}\color{red}\bf
(\%{}i45) 
\end{minipage}
\begin{minipage}[t]{\textwidth}\color{blue}\tt
I:-'integrate('integrate(integrand,u,0,4),v,0,2*\ensuremath{\pi})\$
\end{minipage}


\noindent
%%%%%%%%%%%%%%%
%%% INPUT:
\begin{minipage}[t]{8ex}\color{red}\bf
(\%{}i46) 
\end{minipage}
\begin{minipage}[t]{\textwidth}\color{blue}\tt
ldisplay(I=box(ev(I,integrate)))\$
\end{minipage}
%%% OUTPUT:
\[\displaystyle
\tag{\%{}t46}\label{t46} 
8\ensuremath{\pi} \int_{0}^{4}{\left. udu\right.}=\left( 64\ensuremath{\pi} \right) \mbox{}
\]
%%%%%%%%%%%%%%%
\end{document}
