\documentclass{article}

%% Created with wxMaxima 16.04.2

\setlength{\parskip}{\medskipamount}
\setlength{\parindent}{0pt}
\usepackage[utf8]{inputenc}
\DeclareUnicodeCharacter{00B5}{\ensuremath{\mu}}
\usepackage{graphicx}
\usepackage{color}
\usepackage{amsmath}
\usepackage{ifthen}
\newsavebox{\picturebox}
\newlength{\pictureboxwidth}
\newlength{\pictureboxheight}
\newcommand{\includeimage}[1]{
    \savebox{\picturebox}{\includegraphics{#1}}
    \settoheight{\pictureboxheight}{\usebox{\picturebox}}
    \settowidth{\pictureboxwidth}{\usebox{\picturebox}}
    \ifthenelse{\lengthtest{\pictureboxwidth > .95\linewidth}}
    {
        \includegraphics[width=.95\linewidth,height=.80\textheight,keepaspectratio]{#1}
    }
    {
        \ifthenelse{\lengthtest{\pictureboxheight>.80\textheight}}
        {
            \includegraphics[width=.95\linewidth,height=.80\textheight,keepaspectratio]{#1}
            
        }
        {
            \includegraphics{#1}
        }
    }
}
\newlength{\thislabelwidth}
\DeclareMathOperator{\abs}{abs}
\usepackage{animate} % This package is required because the wxMaxima configuration option
                      % "Export animations to TeX" was enabled when this file was generated.

\definecolor{labelcolor}{RGB}{100,0,0}

\usepackage{fullpage}
\usepackage{amssymb}
\usepackage{enumerate}
\usepackage[bookmarks=false,pdfstartview={FitH},colorlinks=true,urlcolor=blue]{hyperref}
\usepackage{bookmark}
\usepackage{mathtools}

\begin{document}

\pagebreak{}
{\Huge {\sc Explicit calculation\\\\of spin connection}}
\setcounter{section}{0}
\setcounter{subsection}{0}
\setcounter{figure}{0}


\hypersetup{pdfauthor={Daniel Volinski},
            pdftitle={Spin connection},
            pdfsubject={General Relativity},
            pdfkeywords={StackExchange}}

Based on StackExchange Article
\href{https://physics.stackexchange.com/questions/530257/explicit-calculation-of-spin-connection-through-cartans-first-structure-equatio}
{Explicit calculation of spin connection through Cartan's first structure equation}

Written by Daniel Volinski at \href{mailto:danielvolinski@yahoo.es}{danielvolinski@yahoo.es}



\noindent
%%%%%%%%%%%%%%%
%%% INPUT:
\begin{minipage}[t]{8ex}\color{red}\bf
(\%{}i2) 
\end{minipage}
\begin{minipage}[t]{\textwidth}\color{blue}\tt
info:build\_info()\$info\ensuremath{@}version;
\end{minipage}
%%% OUTPUT:
\[\displaystyle
\tag{\%{}o2}\label{o2} 
\mbox{}
\]5.38.1



\noindent
%%%%%%%%%%%%%%%
%%% INPUT:
\begin{minipage}[t]{8ex}\color{red}\bf
(\%{}i2) 
\end{minipage}
\begin{minipage}[t]{\textwidth}\color{blue}\tt
reset()\$kill(all)\$
\end{minipage}


\noindent
%%%%%%%%%%%%%%%
%%% INPUT:
\begin{minipage}[t]{8ex}\color{red}\bf
(\%{}i1) 
\end{minipage}
\begin{minipage}[t]{\textwidth}\color{blue}\tt
derivabbrev:true\$
\end{minipage}


\noindent
%%%%%%%%%%%%%%%
%%% INPUT:
\begin{minipage}[t]{8ex}\color{red}\bf
(\%{}i2) 
\end{minipage}
\begin{minipage}[t]{\textwidth}\color{blue}\tt
ratprint:false\$
\end{minipage}


\noindent
%%%%%%%%%%%%%%%
%%% INPUT:
\begin{minipage}[t]{8ex}\color{red}\bf
(\%{}i3) 
\end{minipage}
\begin{minipage}[t]{\textwidth}\color{blue}\tt
fpprintprec:5\$
\end{minipage}


\noindent
%%%%%%%%%%%%%%%
%%% INPUT:
\begin{minipage}[t]{8ex}\color{red}\bf
(\%{}i4) 
\end{minipage}
\begin{minipage}[t]{\textwidth}\color{blue}\tt
if get('vect,'version)=false then load(vect)\$
\end{minipage}


\noindent
%%%%%%%%%%%%%%%
%%% INPUT:
\begin{minipage}[t]{8ex}\color{red}\bf
(\%{}i5) 
\end{minipage}
\begin{minipage}[t]{\textwidth}\color{blue}\tt
if get('cartan,'version)=false then load(cartan)\$
\end{minipage}


\noindent
%%%%%%%%%%%%%%%
%%% INPUT:
\begin{minipage}[t]{8ex}\color{red}\bf
(\%{}i6) 
\end{minipage}
\begin{minipage}[t]{\textwidth}\color{blue}\tt
if get('format,'version)=false then load(format)\$
\end{minipage}


\noindent
%%%%%%%%%%%%%%%
%%% INPUT:
\begin{minipage}[t]{8ex}\color{red}\bf
(\%{}i7) 
\end{minipage}
\begin{minipage}[t]{\textwidth}\color{blue}\tt
norm(u):=block(ratsimp(radcan(\ensuremath{\sqrt{}}(u.u))))\$
\end{minipage}


\noindent
%%%%%%%%%%%%%%%
%%% INPUT:
\begin{minipage}[t]{8ex}\color{red}\bf
(\%{}i8) 
\end{minipage}
\begin{minipage}[t]{\textwidth}\color{blue}\tt
normalize(v):=block(v/norm(v))\$
\end{minipage}


\noindent
%%%%%%%%%%%%%%%
%%% INPUT:
\begin{minipage}[t]{8ex}\color{red}\bf
(\%{}i9) 
\end{minipage}
\begin{minipage}[t]{\textwidth}\color{blue}\tt
angle(u,v):=block([junk:radcan(\ensuremath{\sqrt{}}((u.u)*(v.v)))],acos(u.v/junk))\$
\end{minipage}


\noindent
%%%%%%%%%%%%%%%
%%% INPUT:
\begin{minipage}[t]{8ex}\color{red}\bf
(\%{}i10) 
\end{minipage}
\begin{minipage}[t]{\textwidth}\color{blue}\tt
mycross(va,vb):=[va[2]*vb[3]-va[3]*vb[2],va[3]*vb[1]-va[1]*vb[3],va[1]*vb[2]-va[2]*vb[1]]\$
\end{minipage}


\noindent
%%%%%%%%%%%%%%%
%%% INPUT:
\begin{minipage}[t]{8ex}\color{red}\bf
(\%{}i11) 
\end{minipage}
\begin{minipage}[t]{\textwidth}\color{blue}\tt
declare(trigsimp,evfun)\$
\end{minipage}
\pagebreak


\section{Jesse wrote:}


Given the metric
\begin{equation}
\mathrm{d}s^2 = F(r)^2\,\mathrm{d}r^2 +
r^2\,\mathrm{d}\theta^2 +
r^2\sin^2(\theta)\,\mathrm{d}\phi^2,
\notag
\end{equation}

I'm trying to find the corresponding spin connections
$\omega^a_{\ b}$ using the first structure equation:

\begin{equation}
\mathrm{d}e + \omega e = 0
\notag
\end{equation}

I found the vielbeins $e$ and their exterior derivatives $\mathrm{d}e$:

\begin{equation}
\mathrm{d}e^1 = 0, \quad
\mathrm{d}e^2 =\mathrm{d}r\mathrm{d}\theta, \quad
\mathrm{d}e^3 = \sin(\theta)\mathrm{d}r\mathrm{d}\phi +
r\cos(\theta)\mathrm{d}\theta \mathrm{d}\phi,
\notag
\end{equation}

but I am stuck on actually working out the $\omega$.
I read through Zee's 'GR in a nutshell', and he does the
same calculation but just says: "In general, write
 
\begin{equation}
\omega^a_{\,b} = \omega^a_{\,bc}e^c = \omega^a_{\,b\mu}dx^\mu
\notag
\end{equation}

Plug this into the first structure equation and match terms.
How do I actually go about calculating $\omega^1_{\, 2}$,
$\omega^1_{\, 3}$, and $\omega^2_{\, 3}$ at this point?



\noindent
%%%%%%%%%%%%%%%
%%% INPUT:
\begin{minipage}[t]{8ex}\color{red}\bf
(\%{}i15) 
\end{minipage}
\begin{minipage}[t]{\textwidth}\color{blue}\tt
assume(0\ensuremath{\leq}r)\$\\
assume(0\ensuremath{\leq}\ensuremath{\theta},\ensuremath{\theta}\ensuremath{\leq}\ensuremath{\pi})\$\\
assume(0\ensuremath{\leq}sin(\ensuremath{\theta}))\$\\
assume(0\ensuremath{\leq}\ensuremath{\phi},\ensuremath{\phi}\ensuremath{\leq}2*\ensuremath{\pi})\$
\end{minipage}


\noindent
%%%%%%%%%%%%%%%
%%% INPUT:
\begin{minipage}[t]{8ex}\color{red}\bf
(\%{}i16) 
\end{minipage}
\begin{minipage}[t]{\textwidth}\color{blue}\tt
\ensuremath{\xi}:[r,\ensuremath{\theta},\ensuremath{\phi}]\$
\end{minipage}


\noindent
%%%%%%%%%%%%%%%
%%% INPUT:
\begin{minipage}[t]{8ex}\color{red}\bf
(\%{}i17) 
\end{minipage}
\begin{minipage}[t]{\textwidth}\color{blue}\tt
dim:length(\ensuremath{\xi})\$
\end{minipage}

\textbf{Line Element}



\noindent
%%%%%%%%%%%%%%%
%%% INPUT:
\begin{minipage}[t]{8ex}\color{red}\bf
(\%{}i18) 
\end{minipage}
\begin{minipage}[t]{\textwidth}\color{blue}\tt
depends(F,r)\$
\end{minipage}


\noindent
%%%%%%%%%%%%%%%
%%% INPUT:
\begin{minipage}[t]{8ex}\color{red}\bf
(\%{}i19) 
\end{minipage}
\begin{minipage}[t]{\textwidth}\color{blue}\tt
assume(F\ensuremath{>}0)\$
\end{minipage}


\noindent
%%%%%%%%%%%%%%%
%%% INPUT:
\begin{minipage}[t]{8ex}\color{red}\bf
(\%{}i20) 
\end{minipage}
\begin{minipage}[t]{\textwidth}\color{blue}\tt
ldisplay(ds\ensuremath{^2}=line\_element:F\ensuremath{^2}*del(r)\ensuremath{^2}+r\ensuremath{^2}*del(\ensuremath{\theta})\ensuremath{^2}+r\ensuremath{^2}*sin(\ensuremath{\theta})\ensuremath{^2}*del(\ensuremath{\phi})\ensuremath{^2})\$
\end{minipage}
%%% OUTPUT:
\[\displaystyle
\tag{\%{}t20}\label{t20} 
{{\mathit{ds}}^{2}}={{r}^{2}}\,{{\sin{\left( \mathit{\ensuremath{\theta}}\right) }}^{2}}\,{{\operatorname{del}\left( \mathit{\ensuremath{\phi}}\right) }^{2}}+{{r}^{2}}\,{{\operatorname{del}\left( \mathit{\ensuremath{\theta}}\right) }^{2}}+{{F}^{2}}\,{{\operatorname{del}(r)}^{2}}\mbox{}
\]
%%%%%%%%%%%%%%%

\textbf{Covariant Metric Tensor}



\noindent
%%%%%%%%%%%%%%%
%%% INPUT:
\begin{minipage}[t]{8ex}\color{red}\bf
(\%{}i24) 
\end{minipage}
\begin{minipage}[t]{\textwidth}\color{blue}\tt
lg:zeromatrix(dim,dim)\$\\
for i thru dim do\\
  lg[i,i]:factor(coeff(expand(line\_element),del(\ensuremath{\xi}[i])\ensuremath{^2}))\$\\
for j thru dim do for k thru dim do\\
  if j\ensuremath{\neq}k then lg[j,k]:factor(expand(ratsimp(\ensuremath{\frac{1}{2}}*coeff(coeff(expand(line\_element),del(\ensuremath{\xi}[j])),del(\ensuremath{\xi}[k])))))\$\\
ldisplay(lg)\$
\end{minipage}
%%% OUTPUT:
\[\displaystyle
\tag{\%{}t24}\label{t24} 
\mathit{lg}=\begin{pmatrix}{{F}^{2}} & 0 & 0\\
0 & {{r}^{2}} & 0\\
0 & 0 & {{r}^{2}}\,{{\sin{\left( \mathit{\ensuremath{\theta}}\right) }}^{2}}\end{pmatrix}\mbox{}
\]
%%%%%%%%%%%%%%%

\textbf{Contravariant Metric Tensor}



\noindent
%%%%%%%%%%%%%%%
%%% INPUT:
\begin{minipage}[t]{8ex}\color{red}\bf
(\%{}i25) 
\end{minipage}
\begin{minipage}[t]{\textwidth}\color{blue}\tt
ldisplay(ug:invert(lg))\$
\end{minipage}
%%% OUTPUT:
\[\displaystyle
\tag{\%{}t25}\label{t25} 
\mathit{ug}=\begin{pmatrix}\frac{1}{{{F}^{2}}} & 0 & 0\\
0 & \frac{1}{{{r}^{2}}} & 0\\
0 & 0 & \frac{1}{{{r}^{2}}\,{{\sin{\left( \mathit{\ensuremath{\theta}}\right) }}^{2}}}\end{pmatrix}\mbox{}
\]
%%%%%%%%%%%%%%%

\textbf{Line element}



\noindent
%%%%%%%%%%%%%%%
%%% INPUT:
\begin{minipage}[t]{8ex}\color{red}\bf
(\%{}i26) 
\end{minipage}
\begin{minipage}[t]{\textwidth}\color{blue}\tt
ldisplay(ds\ensuremath{^2}=diff(\ensuremath{\xi}).lg.transpose(diff(\ensuremath{\xi})))\$
\end{minipage}
%%% OUTPUT:
\[\displaystyle
\tag{\%{}t26}\label{t26} 
{{\mathit{ds}}^{2}}={{r}^{2}}\,{{\sin{\left( \mathit{\ensuremath{\theta}}\right) }}^{2}}\,{{\operatorname{del}\left( \mathit{\ensuremath{\phi}}\right) }^{2}}+{{r}^{2}}\,{{\operatorname{del}\left( \mathit{\ensuremath{\theta}}\right) }^{2}}+{{F}^{2}}\,{{\operatorname{del}(r)}^{2}}\mbox{}
\]
%%%%%%%%%%%%%%%

\textbf{Define the frame} $e$



\noindent
%%%%%%%%%%%%%%%
%%% INPUT:
\begin{minipage}[t]{8ex}\color{red}\bf
(\%{}i29) 
\end{minipage}
\begin{minipage}[t]{\textwidth}\color{blue}\tt
e[r]:\ensuremath{\sqrt{}}(ug)[1]\$\\
e[\ensuremath{\theta}]:\ensuremath{\sqrt{}}(ug)[2]\$\\
e[\ensuremath{\phi}]:\ensuremath{\sqrt{}}(ug)[3]\$
\end{minipage}


\noindent
%%%%%%%%%%%%%%%
%%% INPUT:
\begin{minipage}[t]{8ex}\color{red}\bf
(\%{}i30) 
\end{minipage}
\begin{minipage}[t]{\textwidth}\color{blue}\tt
ldisplay(e:apply('matrix,[e[r],e[\ensuremath{\theta}],e[\ensuremath{\phi}]]))\$
\end{minipage}
%%% OUTPUT:
\[\displaystyle
\tag{\%{}t30}\label{t30} 
e=\begin{pmatrix}\frac{1}{F} & 0 & 0\\
0 & \frac{1}{r} & 0\\
0 & 0 & \frac{1}{r\,\sin{\left( \mathit{\ensuremath{\theta}}\right) }}\end{pmatrix}\mbox{}
\]
%%%%%%%%%%%%%%%

\textbf{Initialize cartan package}



\noindent
%%%%%%%%%%%%%%%
%%% INPUT:
\begin{minipage}[t]{8ex}\color{red}\bf
(\%{}i31) 
\end{minipage}
\begin{minipage}[t]{\textwidth}\color{blue}\tt
init\_cartan(\ensuremath{\xi})\$
\end{minipage}


\noindent
%%%%%%%%%%%%%%%
%%% INPUT:
\begin{minipage}[t]{8ex}\color{red}\bf
(\%{}i32) 
\end{minipage}
\begin{minipage}[t]{\textwidth}\color{blue}\tt
cartan\_basis;
\end{minipage}
%%% OUTPUT:
\[\displaystyle
\tag{\%{}o32}\label{o32} 
[\mathit{dr},\mathit{d\ensuremath{\theta}},\mathit{d\ensuremath{\phi}}]\mbox{}
\]
%%%%%%%%%%%%%%%


\noindent
%%%%%%%%%%%%%%%
%%% INPUT:
\begin{minipage}[t]{8ex}\color{red}\bf
(\%{}i33) 
\end{minipage}
\begin{minipage}[t]{\textwidth}\color{blue}\tt
cartan\_coords;
\end{minipage}
%%% OUTPUT:
\[\displaystyle
\tag{\%{}o33}\label{o33} 
[r,\mathit{\ensuremath{\theta}},\mathit{\ensuremath{\phi}}]\mbox{}
\]
%%%%%%%%%%%%%%%


\noindent
%%%%%%%%%%%%%%%
%%% INPUT:
\begin{minipage}[t]{8ex}\color{red}\bf
(\%{}i34) 
\end{minipage}
\begin{minipage}[t]{\textwidth}\color{blue}\tt
cartan\_dim;
\end{minipage}
%%% OUTPUT:
\[\displaystyle
\tag{\%{}o34}\label{o34} 
3\mbox{}
\]
%%%%%%%%%%%%%%%


\noindent
%%%%%%%%%%%%%%%
%%% INPUT:
\begin{minipage}[t]{8ex}\color{red}\bf
(\%{}i35) 
\end{minipage}
\begin{minipage}[t]{\textwidth}\color{blue}\tt
extdim;
\end{minipage}
%%% OUTPUT:
\[\displaystyle
\tag{\%{}o35}\label{o35} 
3\mbox{}
\]
%%%%%%%%%%%%%%%

\pagebreak
\textbf{Define the coframe} $\omega$



\noindent
%%%%%%%%%%%%%%%
%%% INPUT:
\begin{minipage}[t]{8ex}\color{red}\bf
(\%{}i40) 
\end{minipage}
\begin{minipage}[t]{\textwidth}\color{blue}\tt
kill(\ensuremath{\omega})\$\\
\ensuremath{\omega}[r]:list\_matrix\_entries(\ensuremath{\sqrt{}}(lg).cartan\_basis)[1]\$\\
\ensuremath{\omega}[\ensuremath{\theta}]:list\_matrix\_entries(\ensuremath{\sqrt{}}(lg).cartan\_basis)[2]\$\\
\ensuremath{\omega}[\ensuremath{\phi}]:list\_matrix\_entries(\ensuremath{\sqrt{}}(lg).cartan\_basis)[3]\$\\
ldisplay(\ensuremath{\omega}:[\ensuremath{\omega}[r],\ensuremath{\omega}[\ensuremath{\theta}],\ensuremath{\omega}[\ensuremath{\phi}]])\$
\end{minipage}
%%% OUTPUT:
\[\displaystyle
\tag{\%{}t40}\label{t40} 
\mathit{\ensuremath{\omega}}=[F\,\mathit{dr},r\,\mathit{d\ensuremath{\theta}},r\,\mathit{d\ensuremath{\phi}}\,\sin{\left( \mathit{\ensuremath{\theta}}\right) }]\mbox{}
\]
%%%%%%%%%%%%%%%

\textbf{Verify} $\langle\underline{\omega}^{\color{red} a}
\mid \underline{e}_{\color{red} b}\rangle=
\delta^{\color{red} a}_{\color{red} b}$



\noindent
%%%%%%%%%%%%%%%
%%% INPUT:
\begin{minipage}[t]{8ex}\color{red}\bf
(\%{}i41) 
\end{minipage}
\begin{minipage}[t]{\textwidth}\color{blue}\tt
genmatrix(lambda([i,j],e[\ensuremath{\xi}[i]]|\ensuremath{\omega}[\ensuremath{\xi}[j]]),cartan\_dim,cartan\_dim);
\end{minipage}
%%% OUTPUT:
\[\displaystyle
\tag{\%{}o41}\label{o41} 
\begin{pmatrix}1 & 0 & 0\\
0 & 1 & 0\\
0 & 0 & 1\end{pmatrix}\mbox{}
\]
%%%%%%%%%%%%%%%

\textbf{Calculate the external derivative of the coframe} $\mathrm{d}\omega$



\noindent
%%%%%%%%%%%%%%%
%%% INPUT:
\begin{minipage}[t]{8ex}\color{red}\bf
(\%{}i42) 
\end{minipage}
\begin{minipage}[t]{\textwidth}\color{blue}\tt
ldisplay(d\ensuremath{\omega}:ext\_diff(\ensuremath{\omega}))\$
\end{minipage}
%%% OUTPUT:
\[\displaystyle
\tag{\%{}t42}\label{t42} 
\mathit{d\ensuremath{\omega}}=[0,\mathit{dr}\,\mathit{d\ensuremath{\theta}},\mathit{dr}\,\mathit{d\ensuremath{\phi}}\,\sin{\left( \mathit{\ensuremath{\theta}}\right) }+r\,\mathit{d\ensuremath{\theta}}\,\mathit{d\ensuremath{\phi}}\,\cos{\left( \mathit{\ensuremath{\theta}}\right) }]\mbox{}
\]
%%%%%%%%%%%%%%%
\pagebreak


\section{JamalS wrote:}


As you have, the first step is to identify:

\begin{equation}
e^r = F(r)\,\mathrm{d}r,
e^\theta = r\,\mathrm{d}\theta,
e^\phi = r\sin\theta\,\mathrm{d}\phi
\notag
\end{equation}

The trick is to then take the derivatives but re-express
them in terms of $e$ again. Thus,

\begin{equation}
\mathrm{d}e^r = 0,\quad
\mathrm{d}e^\theta = -\mathrm{d\theta} \wedge
\mathrm{d}r = -\frac{1}{rF(r)}e^\theta \wedge e^r
\notag
\end{equation}

and,

\begin{equation}
\mathrm{de^\phi} = -\sin\theta \mathrm{d\phi} \wedge
\mathrm{d}r - r\cos\theta \mathrm{d}\phi \wedge
\mathrm{d}\theta = -\frac{1}{rF(r)} e^\phi \wedge
e^r - \frac{\cot \theta}{r^2} e^\phi \wedge e^\theta.
\notag
\end{equation}

Now let's take an example of using Cartan's first equation.
We have $\mathrm{d}e^a + \omega^a_b \wedge e^b = 0$
and if we choose $a=\theta$ the equations read,

\begin{equation}
\frac{1}{rF(r)}e^\theta \wedge e^r = \omega^\theta_r
\wedge e^r + \omega^\theta_\theta \wedge e^\theta +
\omega^\theta_\phi \wedge e^\phi.
\notag
\end{equation}

We have $\omega^\theta_\theta = 0$ by anti-symmetry.
We can identify now $\omega^\theta_r = -\omega^r_\theta =
\frac{1}{rF(r)}e^\theta$. Notice the last term we could
choose $\omega^\theta_\phi = 0$ however Cartan's equations
are a system of equations, so we are not free to make this
choice yet without considering the other equations. We can
at best say $\omega^\theta_\phi$ is proportional to
$\mathrm{d}\phi$ to ensure $\omega^\theta_\phi \wedge
e^\phi = 0$. As it turns out, we don't have
$\omega^\theta_\phi = 0$ because of the $a=\phi$ equation,
which will give you $\omega^\theta_\phi = -r^{-2}\cot\theta
\, e^\phi$.

I hope this elucidates how to use Cartan's structure
equation. Computing the Ricci tensor is then much simpler,
as rather than solving for components you're just plugging
in and computing.


\textbf{Generic Connection 1-form} $\Theta$



\noindent
%%%%%%%%%%%%%%%
%%% INPUT:
\begin{minipage}[t]{8ex}\color{red}\bf
(\%{}i45) 
\end{minipage}
\begin{minipage}[t]{\textwidth}\color{blue}\tt
A:[a\_1,a\_2,a\_3]\$\\
B:[b\_1,b\_2,b\_3]\$\\
C:[c\_1,c\_2,c\_3]\$
\end{minipage}


\noindent
%%%%%%%%%%%%%%%
%%% INPUT:
\begin{minipage}[t]{8ex}\color{red}\bf
(\%{}i51) 
\end{minipage}
\begin{minipage}[t]{\textwidth}\color{blue}\tt
kill(\ensuremath{\Theta})\$\\
\ensuremath{\Theta}:zeromatrix(dim,dim)\$\\
\ensuremath{\Theta}[1,2]:-\ensuremath{\Theta}[2,1]:A.cartan\_basis\$\\
\ensuremath{\Theta}[1,3]:-\ensuremath{\Theta}[3,1]:B.cartan\_basis\$\\
\ensuremath{\Theta}[2,3]:-\ensuremath{\Theta}[3,2]:C.cartan\_basis\$\\
ldisplay(\ensuremath{\Theta})\$
\end{minipage}
%%% OUTPUT:
\[\displaystyle
\tag{\%{}t51}\label{t51} 
\mathit{\ensuremath{\Theta}}=\begin{pmatrix}0 & -{{a}_{3}}\mathit{d\ensuremath{\phi}}-{{a}_{2}}\mathit{d\ensuremath{\theta}}-{{a}_{1}}\mathit{dr} & -{{b}_{3}}\mathit{d\ensuremath{\phi}}-{{b}_{2}}\mathit{d\ensuremath{\theta}}-{{b}_{1}}\mathit{dr}\\
{{a}_{3}}\mathit{d\ensuremath{\phi}}+{{a}_{2}}\mathit{d\ensuremath{\theta}}+{{a}_{1}}\mathit{dr} & 0 & -{{c}_{3}}\mathit{d\ensuremath{\phi}}-{{c}_{2}}\mathit{d\ensuremath{\theta}}-{{c}_{1}}\mathit{dr}\\
{{b}_{3}}\mathit{d\ensuremath{\phi}}+{{b}_{2}}\mathit{d\ensuremath{\theta}}+{{b}_{1}}\mathit{dr} & {{c}_{3}}\mathit{d\ensuremath{\phi}}+{{c}_{2}}\mathit{d\ensuremath{\theta}}+{{c}_{1}}\mathit{dr} & 0\end{pmatrix}\mbox{}
\]
%%%%%%%%%%%%%%%

\textbf{Change matrix multiplication operator}



\noindent
%%%%%%%%%%%%%%%
%%% INPUT:
\begin{minipage}[t]{8ex}\color{red}\bf
(\%{}i52) 
\end{minipage}
\begin{minipage}[t]{\textwidth}\color{blue}\tt
matrix\_element\_mult:"\ensuremath{\sim }"\$
\end{minipage}


\noindent
%%%%%%%%%%%%%%%
%%% INPUT:
\begin{minipage}[t]{8ex}\color{red}\bf
(\%{}i53) 
\end{minipage}
\begin{minipage}[t]{\textwidth}\color{blue}\tt
\ensuremath{\lambda}:list\_matrix\_entries(expand(\ensuremath{\Theta}.\ensuremath{\omega}))\$
\end{minipage}


\noindent
%%%%%%%%%%%%%%%
%%% INPUT:
\begin{minipage}[t]{8ex}\color{red}\bf
(\%{}i54) 
\end{minipage}
\begin{minipage}[t]{\textwidth}\color{blue}\tt
map(ldisp,\ensuremath{\lambda})\$
\end{minipage}
%%% OUTPUT:
\[\displaystyle
\tag{\%{}t54}\label{t54} 
-{{b}_{2}}r\,\mathit{d\ensuremath{\theta}}\,\mathit{d\ensuremath{\phi}}\,\sin{\left( \mathit{\ensuremath{\theta}}\right) }-{{b}_{1}}r\,\mathit{dr}\,\mathit{d\ensuremath{\phi}}\,\sin{\left( \mathit{\ensuremath{\theta}}\right) }+{{a}_{3}}r\,\mathit{d\ensuremath{\theta}}\,\mathit{d\ensuremath{\phi}}-{{a}_{1}}r\,\mathit{dr}\,\mathit{d\ensuremath{\theta}}\mbox{}\]
\[\tag{\%{}t55}\label{t55} 
-{{c}_{2}}r\,\mathit{d\ensuremath{\theta}}\,\mathit{d\ensuremath{\phi}}\,\sin{\left( \mathit{\ensuremath{\theta}}\right) }-{{c}_{1}}r\,\mathit{dr}\,\mathit{d\ensuremath{\phi}}\,\sin{\left( \mathit{\ensuremath{\theta}}\right) }-F\,{{a}_{3}}\mathit{dr}\,\mathit{d\ensuremath{\phi}}-F\,{{a}_{2}}\mathit{dr}\,\mathit{d\ensuremath{\theta}}\mbox{}\]
\[\tag{\%{}t56}\label{t56} 
-{{c}_{3}}r\,\mathit{d\ensuremath{\theta}}\,\mathit{d\ensuremath{\phi}}-F\,{{b}_{3}}\mathit{dr}\,\mathit{d\ensuremath{\phi}}+{{c}_{1}}r\,\mathit{dr}\,\mathit{d\ensuremath{\theta}}-F\,{{b}_{2}}\mathit{dr}\,\mathit{d\ensuremath{\theta}}\mbox{}
\]
%%%%%%%%%%%%%%%

\textbf{Restore matrix multiplication operator}



\noindent
%%%%%%%%%%%%%%%
%%% INPUT:
\begin{minipage}[t]{8ex}\color{red}\bf
(\%{}i57) 
\end{minipage}
\begin{minipage}[t]{\textwidth}\color{blue}\tt
matrix\_element\_mult:"*"\$
\end{minipage}

\textbf{Cartan's First structural equation}
$\mathrm{d}\omega^{\color{red} i}=
\Theta_{\color{blue} j}^{\color{red} i}
\wedge\omega^{\color{blue} j}$



\noindent
%%%%%%%%%%%%%%%
%%% INPUT:
\begin{minipage}[t]{8ex}\color{red}\bf
(\%{}i58) 
\end{minipage}
\begin{minipage}[t]{\textwidth}\color{blue}\tt
Eq:zeromatrix(dim,dim)\$
\end{minipage}


\noindent
%%%%%%%%%%%%%%%
%%% INPUT:
\begin{minipage}[t]{8ex}\color{red}\bf
(\%{}i59) 
\end{minipage}
\begin{minipage}[t]{\textwidth}\color{blue}\tt
for i thru dim do for j thru dim do\\
Eq[i,j]:format(coeff(coeff(d\ensuremath{\omega},cartan\_basis[i]),cartan\_basis[j])=\\
coeff(coeff(-\ensuremath{\lambda},cartan\_basis[i]),cartan\_basis[j]),\%list)\$
\end{minipage}


\noindent
%%%%%%%%%%%%%%%
%%% INPUT:
\begin{minipage}[t]{8ex}\color{red}\bf
(\%{}i60) 
\end{minipage}
\begin{minipage}[t]{\textwidth}\color{blue}\tt
Eqs:apply('append,list\_matrix\_entries(Eq))\$
\end{minipage}


\noindent
%%%%%%%%%%%%%%%
%%% INPUT:
\begin{minipage}[t]{8ex}\color{red}\bf
(\%{}i61) 
\end{minipage}
\begin{minipage}[t]{\textwidth}\color{blue}\tt
linsol:linsolve(Eqs,append(A,B,C));
\end{minipage}
%%% OUTPUT:
\[\displaystyle
\mbox{}\\\mbox{solve: dependent equations eliminated: (1 27 26 25 2 3 13 14 15 10 17 19 16 11 12 20 21 18)}\mbox{}\]
\[\tag{linsol}\label{linsol}
\left[{{a}_{1}}=0,{{a}_{2}}=\frac{1}{F},{{a}_{3}}=0,{{b}_{1}}=0,{{b}_{2}}=0,{{b}_{3}}=\frac{\sin{\left( \mathit{\ensuremath{\theta}}\right) }}{F},{{c}_{1}}=0,{{c}_{2}}=0,{{c}_{3}}=\cos{\left( \mathit{\ensuremath{\theta}}\right) }\right]\mbox{}
\]
%%%%%%%%%%%%%%%


\noindent
%%%%%%%%%%%%%%%
%%% INPUT:
\begin{minipage}[t]{8ex}\color{red}\bf
(\%{}i62) 
\end{minipage}
\begin{minipage}[t]{\textwidth}\color{blue}\tt
ldisplay(\ensuremath{\lambda}:at(\ensuremath{\lambda},linsol))\$
\end{minipage}
%%% OUTPUT:
\[\displaystyle
\tag{\%{}t62}\label{t62} 
\mathit{\ensuremath{\lambda}}=[0,-\mathit{dr}\,\mathit{d\ensuremath{\theta}},-\mathit{dr}\,\mathit{d\ensuremath{\phi}}\,\sin{\left( \mathit{\ensuremath{\theta}}\right) }-r\,\mathit{d\ensuremath{\theta}}\,\mathit{d\ensuremath{\phi}}\,\cos{\left( \mathit{\ensuremath{\theta}}\right) }]\mbox{}
\]
%%%%%%%%%%%%%%%


\noindent
%%%%%%%%%%%%%%%
%%% INPUT:
\begin{minipage}[t]{8ex}\color{red}\bf
(\%{}i63) 
\end{minipage}
\begin{minipage}[t]{\textwidth}\color{blue}\tt
is(d\ensuremath{\omega}=-\ensuremath{\lambda});
\end{minipage}
%%% OUTPUT:
\[\displaystyle
\tag{\%{}o63}\label{o63} 
\mbox{true}\mbox{}
\]
%%%%%%%%%%%%%%%

\textbf{Update Connection 1-form} $\Theta$



\noindent
%%%%%%%%%%%%%%%
%%% INPUT:
\begin{minipage}[t]{8ex}\color{red}\bf
(\%{}i64) 
\end{minipage}
\begin{minipage}[t]{\textwidth}\color{blue}\tt
ldisplay(\ensuremath{\Theta}:at(\ensuremath{\Theta},linsol))\$
\end{minipage}
%%% OUTPUT:
\[\displaystyle
\tag{\%{}t64}\label{t64} 
\mathit{\ensuremath{\Theta}}=\begin{pmatrix}0 & -\frac{\mathit{d\ensuremath{\theta}}}{F} & -\frac{\mathit{d\ensuremath{\phi}}\,\sin{\left( \mathit{\ensuremath{\theta}}\right) }}{F}\\
\frac{\mathit{d\ensuremath{\theta}}}{F} & 0 & -\mathit{d\ensuremath{\phi}}\,\cos{\left( \mathit{\ensuremath{\theta}}\right) }\\
\frac{\mathit{d\ensuremath{\phi}}\,\sin{\left( \mathit{\ensuremath{\theta}}\right) }}{F} & \mathit{d\ensuremath{\phi}}\,\cos{\left( \mathit{\ensuremath{\theta}}\right) } & 0\end{pmatrix}\mbox{}
\]
%%%%%%%%%%%%%%%

\textbf{Update Connection 2-form} $\mathrm{d}\Theta$



\noindent
%%%%%%%%%%%%%%%
%%% INPUT:
\begin{minipage}[t]{8ex}\color{red}\bf
(\%{}i65) 
\end{minipage}
\begin{minipage}[t]{\textwidth}\color{blue}\tt
ldisplay(d\ensuremath{\Theta}:expand(matrixmap(edit,ext\_diff(\ensuremath{\Theta}))))\$
\end{minipage}
%%% OUTPUT:
\[\displaystyle
\tag{\%{}t65}\label{t65} 
\mathit{d\ensuremath{\Theta}}=\begin{pmatrix}0 & \frac{\left( {{F}_{r}}\right) \,\mathit{dr}\,\mathit{d\ensuremath{\theta}}}{{{F}^{2}}} & \frac{\left( {{F}_{r}}\right) \,\mathit{dr}\,\mathit{d\ensuremath{\phi}}\,\sin{\left( \mathit{\ensuremath{\theta}}\right) }}{{{F}^{2}}}-\frac{\mathit{d\ensuremath{\theta}}\,\mathit{d\ensuremath{\phi}}\,\cos{\left( \mathit{\ensuremath{\theta}}\right) }}{F}\\
-\frac{\left( {{F}_{r}}\right) \,\mathit{dr}\,\mathit{d\ensuremath{\theta}}}{{{F}^{2}}} & 0 & \mathit{d\ensuremath{\theta}}\,\mathit{d\ensuremath{\phi}}\,\sin{\left( \mathit{\ensuremath{\theta}}\right) }\\
\frac{\mathit{d\ensuremath{\theta}}\,\mathit{d\ensuremath{\phi}}\,\cos{\left( \mathit{\ensuremath{\theta}}\right) }}{F}-\frac{\left( {{F}_{r}}\right) \,\mathit{dr}\,\mathit{d\ensuremath{\phi}}\,\sin{\left( \mathit{\ensuremath{\theta}}\right) }}{{{F}^{2}}} & -\mathit{d\ensuremath{\theta}}\,\mathit{d\ensuremath{\phi}}\,\sin{\left( \mathit{\ensuremath{\theta}}\right) } & 0\end{pmatrix}\mbox{}
\]
%%%%%%%%%%%%%%%

\textbf{Update coefficients}



\noindent
%%%%%%%%%%%%%%%
%%% INPUT:
\begin{minipage}[t]{8ex}\color{red}\bf
(\%{}i68) 
\end{minipage}
\begin{minipage}[t]{\textwidth}\color{blue}\tt
ldisplay(A:at(A,linsol))\$\\
ldisplay(B:at(B,linsol))\$\\
ldisplay(C:at(C,linsol))\$
\end{minipage}
%%% OUTPUT:
\[\displaystyle
\tag{\%{}t66}\label{t66} 
A=\left[0,\frac{1}{F},0\right]\mbox{}\]
\[\tag{\%{}t67}\label{t67} 
B=\left[0,0,\frac{\sin{\left( \mathit{\ensuremath{\theta}}\right) }}{F}\right]\mbox{}\]
\[\tag{\%{}t68}\label{t68} 
C=\left[0,0,\cos{\left( \mathit{\ensuremath{\theta}}\right) }\right]\mbox{}
\]
%%%%%%%%%%%%%%%

\textbf{Change matrix multiplication operator}



\noindent
%%%%%%%%%%%%%%%
%%% INPUT:
\begin{minipage}[t]{8ex}\color{red}\bf
(\%{}i69) 
\end{minipage}
\begin{minipage}[t]{\textwidth}\color{blue}\tt
matrix\_element\_mult:"\ensuremath{\sim }"\$
\end{minipage}

\textbf{Cartan's Second structural equation}:
$\Omega_{\color{red} j}^{\color{red} i}=
\mathrm{d}\Theta_{\color{red} j}^{\color{red} i}+
\Theta_{\color{blue} k}^{\color{red} i}\wedge
\Theta_{\color{red} j}^{\color{blue} k}$


\textbf{Curvature 2-form} $\Omega$



\noindent
%%%%%%%%%%%%%%%
%%% INPUT:
\begin{minipage}[t]{8ex}\color{red}\bf
(\%{}i70) 
\end{minipage}
\begin{minipage}[t]{\textwidth}\color{blue}\tt
ldisplay(\ensuremath{\Omega}:matrixmap(edit,d\ensuremath{\Theta}+\ensuremath{\Theta}.\ensuremath{\Theta}))\$
\end{minipage}
%%% OUTPUT:
\[\displaystyle
\tag{\%{}t70}\label{t70} 
\mathit{\ensuremath{\Omega}}=\begin{pmatrix}0 & \frac{\left( {{F}_{r}}\right) \,\mathit{dr}\,\mathit{d\ensuremath{\theta}}}{{{F}^{2}}} & \frac{\left( {{F}_{r}}\right) \,\mathit{dr}\,\mathit{d\ensuremath{\phi}}\,\sin{\left( \mathit{\ensuremath{\theta}}\right) }}{{{F}^{2}}}\\
-\frac{\left( {{F}_{r}}\right) \,\mathit{dr}\,\mathit{d\ensuremath{\theta}}}{{{F}^{2}}} & 0 & \mathit{d\ensuremath{\theta}}\,\mathit{d\ensuremath{\phi}}\,\left( \sin{\left( \mathit{\ensuremath{\theta}}\right) }-\frac{\sin{\left( \mathit{\ensuremath{\theta}}\right) }}{{{F}^{2}}}\right) \\
-\frac{\left( {{F}_{r}}\right) \,\mathit{dr}\,\mathit{d\ensuremath{\phi}}\,\sin{\left( \mathit{\ensuremath{\theta}}\right) }}{{{F}^{2}}} & \mathit{d\ensuremath{\theta}}\,\mathit{d\ensuremath{\phi}}\,\left( \frac{\sin{\left( \mathit{\ensuremath{\theta}}\right) }}{{{F}^{2}}}-\sin{\left( \mathit{\ensuremath{\theta}}\right) }\right)  & 0\end{pmatrix}\mbox{}
\]
%%%%%%%%%%%%%%%

\textbf{Restore matrix multiplication operator}



\noindent
%%%%%%%%%%%%%%%
%%% INPUT:
\begin{minipage}[t]{8ex}\color{red}\bf
(\%{}i71) 
\end{minipage}
\begin{minipage}[t]{\textwidth}\color{blue}\tt
matrix\_element\_mult:"*"\$
\end{minipage}

\textbf{Forms in terms of the coframe} $\sigma$



\noindent
%%%%%%%%%%%%%%%
%%% INPUT:
\begin{minipage}[t]{8ex}\color{red}\bf
(\%{}i72) 
\end{minipage}
\begin{minipage}[t]{\textwidth}\color{blue}\tt
Eqs:makelist(\ensuremath{\sigma}[\ensuremath{\xi}[i]]=\ensuremath{\omega}[\ensuremath{\xi}[i]],i,1,cartan\_dim);
\end{minipage}
%%% OUTPUT:
\[\displaystyle
\tag{Eqs}\label{Eqs}
[{{\mathit{\ensuremath{\sigma}}}_{r}}=F\,\mathit{dr},{{\mathit{\ensuremath{\sigma}}}_{\mathit{\ensuremath{\theta}}}}=r\,\mathit{d\ensuremath{\theta}},{{\mathit{\ensuremath{\sigma}}}_{\mathit{\ensuremath{\phi}}}}=r\,\mathit{d\ensuremath{\phi}}\,\sin{\left( \mathit{\ensuremath{\theta}}\right) }]\mbox{}
\]
%%%%%%%%%%%%%%%


\noindent
%%%%%%%%%%%%%%%
%%% INPUT:
\begin{minipage}[t]{8ex}\color{red}\bf
(\%{}i73) 
\end{minipage}
\begin{minipage}[t]{\textwidth}\color{blue}\tt
linsol:linsolve(Eqs,cartan\_basis);
\end{minipage}
%%% OUTPUT:
\[\displaystyle
\tag{linsol}\label{linsol}
\left[\mathit{dr}=\frac{{{\mathit{\ensuremath{\sigma}}}_{r}}}{F},\mathit{d\ensuremath{\theta}}=\frac{{{\mathit{\ensuremath{\sigma}}}_{\mathit{\ensuremath{\theta}}}}}{r},\mathit{d\ensuremath{\phi}}=\frac{{{\mathit{\ensuremath{\sigma}}}_{\mathit{\ensuremath{\phi}}}}}{r\,\sin{\left( \mathit{\ensuremath{\theta}}\right) }}\right]\mbox{}
\]
%%%%%%%%%%%%%%%

\textbf{Connection 1-form} $\Theta$



\noindent
%%%%%%%%%%%%%%%
%%% INPUT:
\begin{minipage}[t]{8ex}\color{red}\bf
(\%{}i74) 
\end{minipage}
\begin{minipage}[t]{\textwidth}\color{blue}\tt
ldisplay(\ensuremath{\Theta}:ev(\ensuremath{\Theta},linsol,fullratsimp))\$
\end{minipage}
%%% OUTPUT:
\[\displaystyle
\tag{\%{}t74}\label{t74} 
\mathit{\ensuremath{\Theta}}=\begin{pmatrix}0 & -\frac{{{\mathit{\ensuremath{\sigma}}}_{\mathit{\ensuremath{\theta}}}}}{Fr} & -\frac{{{\mathit{\ensuremath{\sigma}}}_{\mathit{\ensuremath{\phi}}}}}{Fr}\\
\frac{{{\mathit{\ensuremath{\sigma}}}_{\mathit{\ensuremath{\theta}}}}}{Fr} & 0 & -\frac{\cos{\left( \mathit{\ensuremath{\theta}}\right) }\,{{\mathit{\ensuremath{\sigma}}}_{\mathit{\ensuremath{\phi}}}}}{r\,\sin{\left( \mathit{\ensuremath{\theta}}\right) }}\\
\frac{{{\mathit{\ensuremath{\sigma}}}_{\mathit{\ensuremath{\phi}}}}}{Fr} & \frac{\cos{\left( \mathit{\ensuremath{\theta}}\right) }\,{{\mathit{\ensuremath{\sigma}}}_{\mathit{\ensuremath{\phi}}}}}{r\,\sin{\left( \mathit{\ensuremath{\theta}}\right) }} & 0\end{pmatrix}\mbox{}
\]
%%%%%%%%%%%%%%%

\pagebreak
\textbf{Curvature 2-form} $\Omega$



\noindent
%%%%%%%%%%%%%%%
%%% INPUT:
\begin{minipage}[t]{8ex}\color{red}\bf
(\%{}i75) 
\end{minipage}
\begin{minipage}[t]{\textwidth}\color{blue}\tt
ldisplay(\ensuremath{\Omega}:ev(\ensuremath{\Omega},linsol,fullratsimp))\$
\end{minipage}
%%% OUTPUT:
\[\displaystyle
\tag{\%{}t75}\label{t75} 
\mathit{\ensuremath{\Omega}}=\begin{pmatrix}0 & \frac{\left( {{F}_{r}}\right) \,{{\mathit{\ensuremath{\sigma}}}_{r}}\,{{\mathit{\ensuremath{\sigma}}}_{\mathit{\ensuremath{\theta}}}}}{{{F}^{3}}r} & \frac{\left( {{F}_{r}}\right) \,{{\mathit{\ensuremath{\sigma}}}_{r}}\,{{\mathit{\ensuremath{\sigma}}}_{\mathit{\ensuremath{\phi}}}}}{{{F}^{3}}r}\\
-\frac{\left( {{F}_{r}}\right) \,{{\mathit{\ensuremath{\sigma}}}_{r}}\,{{\mathit{\ensuremath{\sigma}}}_{\mathit{\ensuremath{\theta}}}}}{{{F}^{3}}r} & 0 & \frac{\left( {{F}^{2}}-1\right) \,{{\mathit{\ensuremath{\sigma}}}_{\mathit{\ensuremath{\theta}}}}\,{{\mathit{\ensuremath{\sigma}}}_{\mathit{\ensuremath{\phi}}}}}{{{F}^{2}}\,{{r}^{2}}}\\
-\frac{\left( {{F}_{r}}\right) \,{{\mathit{\ensuremath{\sigma}}}_{r}}\,{{\mathit{\ensuremath{\sigma}}}_{\mathit{\ensuremath{\phi}}}}}{{{F}^{3}}r} & -\frac{\left( {{F}^{2}}-1\right) \,{{\mathit{\ensuremath{\sigma}}}_{\mathit{\ensuremath{\theta}}}}\,{{\mathit{\ensuremath{\sigma}}}_{\mathit{\ensuremath{\phi}}}}}{{{F}^{2}}\,{{r}^{2}}} & 0\end{pmatrix}\mbox{}
\]
%%%%%%%%%%%%%%%

\textbf{Clean up}



\noindent
%%%%%%%%%%%%%%%
%%% INPUT:
\begin{minipage}[t]{8ex}\color{red}\bf
(\%{}i79) 
\end{minipage}
\begin{minipage}[t]{\textwidth}\color{blue}\tt
forget(0\ensuremath{\leq}r)\$\\
forget(0\ensuremath{\leq}\ensuremath{\theta},\ensuremath{\theta}\ensuremath{\leq}\ensuremath{\pi})\$\\
forget(0\ensuremath{\leq}sin(\ensuremath{\theta}))\$\\
forget(0\ensuremath{\leq}\ensuremath{\phi},\ensuremath{\phi}\ensuremath{\leq}2*\ensuremath{\pi})\$
\end{minipage}
\pagebreak


\section{Bence Racskó wrote:}


There is also an explicit procedure that is often better
if the vielbein is simple.

We have
\begin{equation}
\mathrm{d}e^a=-\frac{1}{2}C^a_{bc}e^b\wedge e^c,
\notag
\end{equation}

where the $C^a_{bc}$ are the vielbein commutators.
We can invert the first structure equation explicitly as
\begin{equation}
\begin{split}
0=\mathrm de^a+\omega^a_{\ b}\wedge e^b=
-\frac{1}{2}C^a_{bc}e^b\wedge e^c+
\omega^a_{c\ b}e^c\wedge e^b \\
\frac{1}{2}C^a_{bc}e^b\wedge e^c=
\frac{1}{2}\left( \omega^a_{b\ c}-\omega^a_{c\ b}
\right)e^b\wedge e^c,
\end{split}
\notag
\end{equation}

so
\begin{equation}
C^a_{bc}=\omega^a_{b\ c}-\omega^a_{c\ b}.
\notag
\end{equation}

Lowering the index, we get
\begin{equation}
\begin{split}
C_{a,bc}=\omega_{b,ac}-\omega_{c,ab} \\
C_{b,ca}=\omega_{c,ba}-\omega_{a,bc} \\
-C_{c,ab}=-\omega_{a,cb}+\omega_{b,ca},
\end{split}
\notag
\end{equation}

now sum these up:
\begin{equation}
\begin{split}
C_{a,bc}+C_{b,ca}-C_{c,ab}=2\omega_{c,ba} \\
\omega_{c,ab}=\frac{1}{2}\left(C_{c,ab}-C_{a,bc}-C_{b,ca}\right) \\
\omega_{ab}=\frac{1}{2}\left(C_{c,ab}-C_{a,bc}-C_{b,ca}\right)e^c.
\end{split}
\notag
\end{equation}

If the veilbein is simple, then the $\mathrm de^a=
-\frac{1}{2}C^a_{bc}e^b\wedge e^c$ will only involve a
few terms at most, and the spin connection is very easy
to calculate from this.


\section{JamalS wrote:}


Looks like both our methods are relatively equal in terms
of simplicity - if the vielbein is simple, as you said,
in the first method solving the linear system is also
trivial.


\section{Bence Racskó wrote:}


But I guess this is highly subjective. For vielbeins like
the natural FLRW or Schwarzschild vielbeins, I find this
method simpler and easier. But that's me.

\end{document}
