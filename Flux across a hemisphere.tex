\documentclass{article}

%% Created with wxMaxima 16.04.2

\setlength{\parskip}{\medskipamount}
\setlength{\parindent}{0pt}
\usepackage[utf8]{inputenc}
\DeclareUnicodeCharacter{00B5}{\ensuremath{\mu}}
\usepackage{graphicx}
\usepackage{color}
\usepackage{amsmath}
\usepackage{ifthen}
\newsavebox{\picturebox}
\newlength{\pictureboxwidth}
\newlength{\pictureboxheight}
\newcommand{\includeimage}[1]{
    \savebox{\picturebox}{\includegraphics{#1}}
    \settoheight{\pictureboxheight}{\usebox{\picturebox}}
    \settowidth{\pictureboxwidth}{\usebox{\picturebox}}
    \ifthenelse{\lengthtest{\pictureboxwidth > .95\linewidth}}
    {
        \includegraphics[width=.95\linewidth,height=.80\textheight,keepaspectratio]{#1}
    }
    {
        \ifthenelse{\lengthtest{\pictureboxheight>.80\textheight}}
        {
            \includegraphics[width=.95\linewidth,height=.80\textheight,keepaspectratio]{#1}
            
        }
        {
            \includegraphics{#1}
        }
    }
}
\newlength{\thislabelwidth}
\DeclareMathOperator{\abs}{abs}
\usepackage{animate} % This package is required because the wxMaxima configuration option
                      % "Export animations to TeX" was enabled when this file was generated.

\definecolor{labelcolor}{RGB}{100,0,0}

\usepackage{fullpage}
\usepackage{amssymb}
\usepackage{enumerate}
\usepackage[bookmarks=false,pdfstartview={FitH},colorlinks=true,urlcolor=blue]{hyperref}
\usepackage{bookmark}
\usepackage{mathtools}

\begin{document}

\pagebreak{}
{\Huge {\sc Flux across a hemisphere}}
\setcounter{section}{0}
\setcounter{subsection}{0}
\setcounter{figure}{0}


\hypersetup{pdfauthor={Daniel Volinski},
            pdftitle={Flux example},
            pdfsubject={Multivariable Calculus},
            pdfkeywords={Dr. Bevin Maultsby}}

Based on Dr. Bevin Maultsby Playlist
\href{https://www.youtube.com/watch?v=Qw7jyJZK2Gc&list=PLBEl4BT8wUgOqJCBijQxClMIuumgtMinc&index=108}
{Flux across a hemisphere, with and without the Divergence Theorem}

Written by Daniel Volinski at \href{mailto:danielvolinski@yahoo.es}{danielvolinski@yahoo.es}



\noindent
%%%%%%%%%%%%%%%
%%% INPUT:
\begin{minipage}[t]{8ex}\color{red}\bf
(\%{}i2) 
\end{minipage}
\begin{minipage}[t]{\textwidth}\color{blue}\tt
info:build\_info()\$info\ensuremath{@}version;
\end{minipage}
%%% OUTPUT:
\[\displaystyle
\tag{\%{}o2}\label{o2} 
\mbox{}
\]5.38.1



\noindent
%%%%%%%%%%%%%%%
%%% INPUT:
\begin{minipage}[t]{8ex}\color{red}\bf
(\%{}i2) 
\end{minipage}
\begin{minipage}[t]{\textwidth}\color{blue}\tt
reset()\$kill(all)\$
\end{minipage}


\noindent
%%%%%%%%%%%%%%%
%%% INPUT:
\begin{minipage}[t]{8ex}\color{red}\bf
(\%{}i1) 
\end{minipage}
\begin{minipage}[t]{\textwidth}\color{blue}\tt
derivabbrev:true\$
\end{minipage}


\noindent
%%%%%%%%%%%%%%%
%%% INPUT:
\begin{minipage}[t]{8ex}\color{red}\bf
(\%{}i2) 
\end{minipage}
\begin{minipage}[t]{\textwidth}\color{blue}\tt
ratprint:false\$
\end{minipage}


\noindent
%%%%%%%%%%%%%%%
%%% INPUT:
\begin{minipage}[t]{8ex}\color{red}\bf
(\%{}i3) 
\end{minipage}
\begin{minipage}[t]{\textwidth}\color{blue}\tt
fpprintprec:5\$
\end{minipage}


\noindent
%%%%%%%%%%%%%%%
%%% INPUT:
\begin{minipage}[t]{8ex}\color{red}\bf
(\%{}i4) 
\end{minipage}
\begin{minipage}[t]{\textwidth}\color{blue}\tt
load(linearalgebra)\$
\end{minipage}


\noindent
%%%%%%%%%%%%%%%
%%% INPUT:
\begin{minipage}[t]{8ex}\color{red}\bf
(\%{}i5) 
\end{minipage}
\begin{minipage}[t]{\textwidth}\color{blue}\tt
if get('draw,'version)=false then load(draw)\$
\end{minipage}
%%% OUTPUT:
%%%%%%%%%%%%%%%


\noindent
%%%%%%%%%%%%%%%
%%% INPUT:
\begin{minipage}[t]{8ex}\color{red}\bf
(\%{}i6) 
\end{minipage}
\begin{minipage}[t]{\textwidth}\color{blue}\tt
wxplot\_size:[1024,768]\$
\end{minipage}


\noindent
%%%%%%%%%%%%%%%
%%% INPUT:
\begin{minipage}[t]{8ex}\color{red}\bf
(\%{}i7) 
\end{minipage}
\begin{minipage}[t]{\textwidth}\color{blue}\tt
set\_draw\_defaults(xtics=1,ytics=1,ztics=1,xyplane=0,nticks=100,\\
                  xaxis=true,xaxis\_type=dots,xaxis\_width=3,\\
                  yaxis=true,yaxis\_type=dots,yaxis\_width=3,\\
                  zaxis=true,zaxis\_type=dots,zaxis\_width=3,\\
                  background\_color=light\_gray)\$
\end{minipage}


\noindent
%%%%%%%%%%%%%%%
%%% INPUT:
\begin{minipage}[t]{8ex}\color{red}\bf
(\%{}i8) 
\end{minipage}
\begin{minipage}[t]{\textwidth}\color{blue}\tt
if get('vect,'version)=false then load(vect)\$
\end{minipage}


\noindent
%%%%%%%%%%%%%%%
%%% INPUT:
\begin{minipage}[t]{8ex}\color{red}\bf
(\%{}i9) 
\end{minipage}
\begin{minipage}[t]{\textwidth}\color{blue}\tt
norm(u):=block(ratsimp(radcan(\ensuremath{\sqrt{}}(u.u))))\$
\end{minipage}


\noindent
%%%%%%%%%%%%%%%
%%% INPUT:
\begin{minipage}[t]{8ex}\color{red}\bf
(\%{}i10) 
\end{minipage}
\begin{minipage}[t]{\textwidth}\color{blue}\tt
normalize(v):=block(v/norm(v))\$
\end{minipage}


\noindent
%%%%%%%%%%%%%%%
%%% INPUT:
\begin{minipage}[t]{8ex}\color{red}\bf
(\%{}i11) 
\end{minipage}
\begin{minipage}[t]{\textwidth}\color{blue}\tt
angle(u,v):=block([junk:radcan(\ensuremath{\sqrt{}}((u.u)*(v.v)))],acos(u.v/junk))\$
\end{minipage}


\noindent
%%%%%%%%%%%%%%%
%%% INPUT:
\begin{minipage}[t]{8ex}\color{red}\bf
(\%{}i12) 
\end{minipage}
\begin{minipage}[t]{\textwidth}\color{blue}\tt
mycross(va,vb):=[va[2]*vb[3]-va[3]*vb[2],va[3]*vb[1]-va[1]*vb[3],va[1]*vb[2]-va[2]*vb[1]]\$
\end{minipage}


\noindent
%%%%%%%%%%%%%%%
%%% INPUT:
\begin{minipage}[t]{8ex}\color{red}\bf
(\%{}i13) 
\end{minipage}
\begin{minipage}[t]{\textwidth}\color{blue}\tt
if get('cartan,'version)=false then load(cartan)\$
\end{minipage}


\noindent
%%%%%%%%%%%%%%%
%%% INPUT:
\begin{minipage}[t]{8ex}\color{red}\bf
(\%{}i14) 
\end{minipage}
\begin{minipage}[t]{\textwidth}\color{blue}\tt
declare(trigsimp,evfun)\$
\end{minipage}
\pagebreak


Let $M$ be the surface $x^2+y^2+z^2=9$. Using the outward-pointing
normal, find the flux through $M$ for the vector field
$\vec{F}(x,y,z)=\langle{y,x,z}\rangle$.


\textbf{Define the space} $\mathbb{R}^3$



\noindent
%%%%%%%%%%%%%%%
%%% INPUT:
\begin{minipage}[t]{8ex}\color{red}\bf
(\%{}i15) 
\end{minipage}
\begin{minipage}[t]{\textwidth}\color{blue}\tt
\ensuremath{\zeta}:[x,y,z]\$
\end{minipage}


\noindent
%%%%%%%%%%%%%%%
%%% INPUT:
\begin{minipage}[t]{8ex}\color{red}\bf
(\%{}i16) 
\end{minipage}
\begin{minipage}[t]{\textwidth}\color{blue}\tt
dim:length(\ensuremath{\zeta})\$
\end{minipage}


\noindent
%%%%%%%%%%%%%%%
%%% INPUT:
\begin{minipage}[t]{8ex}\color{red}\bf
(\%{}i17) 
\end{minipage}
\begin{minipage}[t]{\textwidth}\color{blue}\tt
scalefactors(\ensuremath{\zeta})\$
\end{minipage}


\noindent
%%%%%%%%%%%%%%%
%%% INPUT:
\begin{minipage}[t]{8ex}\color{red}\bf
(\%{}i18) 
\end{minipage}
\begin{minipage}[t]{\textwidth}\color{blue}\tt
init\_cartan(\ensuremath{\zeta})\$
\end{minipage}

\textbf{Vector field} $\vec{F}\in\mathbb{R}^3$



\noindent
%%%%%%%%%%%%%%%
%%% INPUT:
\begin{minipage}[t]{8ex}\color{red}\bf
(\%{}i19) 
\end{minipage}
\begin{minipage}[t]{\textwidth}\color{blue}\tt
ldisplay(F:[y,x,z])\$
\end{minipage}
%%% OUTPUT:
\[\displaystyle
\tag{\%{}t19}\label{t19} 
F=[y,x,z]\mbox{}
\]
%%%%%%%%%%%%%%%

\textbf{3D Direction field}



\noindent
%%%%%%%%%%%%%%%
%%% INPUT:
\begin{minipage}[t]{8ex}\color{red}\bf
(\%{}i21) 
\end{minipage}
\begin{minipage}[t]{\textwidth}\color{blue}\tt
/* vector origins are {(x,y,z)| x,y=1,...,5}  */\\
coord:setify(makelist(k,k,-3,3))\$\\
points3d:listify(cartesian\_product(coord,coord,coord))\$
\end{minipage}


\noindent
%%%%%%%%%%%%%%%
%%% INPUT:
\begin{minipage}[t]{8ex}\color{red}\bf
(\%{}i23) 
\end{minipage}
\begin{minipage}[t]{\textwidth}\color{blue}\tt
/* compute vectors at the given points  */\\
define(vf3d(x,y,z),vector(\ensuremath{\zeta},F))\$\\
vect3:makelist(vf3d(k[1],k[2],k[3]),k,points3d)\$
\end{minipage}


\noindent
%%%%%%%%%%%%%%%
%%% INPUT:
\begin{minipage}[t]{8ex}\color{red}\bf
(\%{}i24) 
\end{minipage}
\begin{minipage}[t]{\textwidth}\color{blue}\tt
wxdraw3d([head\_length=0.1,color=blue,head\_angle=25,unit\_vectors=true],vect3)\$
\end{minipage}
%%% OUTPUT:
\[\displaystyle
\tag{\%{}t24}\label{t24} 
\includegraphics[width=.95\linewidth,height=.80\textheight,keepaspectratio]{Flux across a hemisphere_img/Flux across a hemisphere_1}\mbox{}
\]
%%%%%%%%%%%%%%%
\pagebreak


\textbf{Calculate} $\nabla\times\vec{F}\in\mathbb{R}^3$



\noindent
%%%%%%%%%%%%%%%
%%% INPUT:
\begin{minipage}[t]{8ex}\color{red}\bf
(\%{}i25) 
\end{minipage}
\begin{minipage}[t]{\textwidth}\color{blue}\tt
ldisplay(curlF:ev(express(curl(F)),diff))\$
\end{minipage}
%%% OUTPUT:
\[\displaystyle
\tag{\%{}t25}\label{t25} 
\mathit{curlF}=[0,0,0]\mbox{}
\]
%%%%%%%%%%%%%%%

\textbf{Work form} $\alpha=F^\flat\in\mathcal{A}^1(\mathbb{R}^3)$



\noindent
%%%%%%%%%%%%%%%
%%% INPUT:
\begin{minipage}[t]{8ex}\color{red}\bf
(\%{}i26) 
\end{minipage}
\begin{minipage}[t]{\textwidth}\color{blue}\tt
ldisplay(\ensuremath{\alpha}:F.cartan\_basis)\$
\end{minipage}
%%% OUTPUT:
\[\displaystyle
\tag{\%{}t26}\label{t26} 
\mathit{\ensuremath{\alpha}}=z\,\mathit{dz}+x\,\mathit{dy}+y\,\mathit{dx}\mbox{}
\]
%%%%%%%%%%%%%%%

\textbf{Calculate} $\mathrm{d}\alpha\in\mathcal{A}^2(\mathbb{R}^3)$



\noindent
%%%%%%%%%%%%%%%
%%% INPUT:
\begin{minipage}[t]{8ex}\color{red}\bf
(\%{}i27) 
\end{minipage}
\begin{minipage}[t]{\textwidth}\color{blue}\tt
ldisplay(d\ensuremath{\alpha}:ext\_diff(\ensuremath{\alpha}))\$
\end{minipage}
%%% OUTPUT:
\[\displaystyle
\tag{\%{}t27}\label{t27} 
\mathit{d\ensuremath{\alpha}}=0\mbox{}
\]
%%%%%%%%%%%%%%%

\textbf{Calculate} $\nabla\cdot\vec{F}\in\mathbb{R}$



\noindent
%%%%%%%%%%%%%%%
%%% INPUT:
\begin{minipage}[t]{8ex}\color{red}\bf
(\%{}i28) 
\end{minipage}
\begin{minipage}[t]{\textwidth}\color{blue}\tt
ldisplay(divF:ev(express(div(F)),diff))\$
\end{minipage}
%%% OUTPUT:
\[\displaystyle
\tag{\%{}t28}\label{t28} 
\mathit{divF}=1\mbox{}
\]
%%%%%%%%%%%%%%%

\textbf{Flux form} $\beta\in\mathcal{A}^2(\mathbb{R}^3)$



\noindent
%%%%%%%%%%%%%%%
%%% INPUT:
\begin{minipage}[t]{8ex}\color{red}\bf
(\%{}i29) 
\end{minipage}
\begin{minipage}[t]{\textwidth}\color{blue}\tt
ldisplay(\ensuremath{\beta}:F[1]*cartan\_basis[2]\ensuremath{\sim }cartan\_basis[3]+\\
           F[2]*cartan\_basis[3]\ensuremath{\sim }cartan\_basis[1]+\\
           F[3]*cartan\_basis[1]\ensuremath{\sim }cartan\_basis[2])\$
\end{minipage}
%%% OUTPUT:
\[\displaystyle
\tag{\%{}t29}\label{t29} 
\mathit{\ensuremath{\beta}}=y\,\mathit{dy}\,\mathit{dz}-x\,\mathit{dx}\,\mathit{dz}+z\,\mathit{dx}\,\mathit{dy}\mbox{}
\]
%%%%%%%%%%%%%%%


\noindent
%%%%%%%%%%%%%%%
%%% INPUT:
\begin{minipage}[t]{8ex}\color{red}\bf
(\%{}i30) 
\end{minipage}
\begin{minipage}[t]{\textwidth}\color{blue}\tt
\ensuremath{\epsilon}[i,j,k]:=\ensuremath{\frac{1}{2}}*(i-j)*(j-k)*(k-i)\$
\end{minipage}


\noindent
%%%%%%%%%%%%%%%
%%% INPUT:
\begin{minipage}[t]{8ex}\color{red}\bf
(\%{}i31) 
\end{minipage}
\begin{minipage}[t]{\textwidth}\color{blue}\tt
ldisplay(p:edit(\ensuremath{\frac{1}{2}}*sum(sum(sum(\ensuremath{\epsilon}[i,j,k]*\\
           F[i]*cartan\_basis[j]\ensuremath{\sim }cartan\_basis[k],\\
           i,1,dim),j,1,dim),k,1,dim)))\$
\end{minipage}
%%% OUTPUT:
\[\displaystyle
\tag{\%{}t31}\label{t31} 
p=y\,\mathit{dy}\,\mathit{dz}-x\,\mathit{dx}\,\mathit{dz}+z\,\mathit{dx}\,\mathit{dy}\mbox{}
\]
%%%%%%%%%%%%%%%


\noindent
%%%%%%%%%%%%%%%
%%% INPUT:
\begin{minipage}[t]{8ex}\color{red}\bf
(\%{}i32) 
\end{minipage}
\begin{minipage}[t]{\textwidth}\color{blue}\tt
is(p=\ensuremath{\beta});
\end{minipage}
%%% OUTPUT:
\[\displaystyle
\tag{\%{}o32}\label{o32} 
\mbox{true}\mbox{}
\]
%%%%%%%%%%%%%%%

\textbf{Calculate} $\mathrm{d}\beta\in\mathcal{A}^3(\mathbb{R}^3)$



\noindent
%%%%%%%%%%%%%%%
%%% INPUT:
\begin{minipage}[t]{8ex}\color{red}\bf
(\%{}i33) 
\end{minipage}
\begin{minipage}[t]{\textwidth}\color{blue}\tt
ldisplay(d\ensuremath{\beta}:ext\_diff(\ensuremath{\beta}))\$
\end{minipage}
%%% OUTPUT:
\[\displaystyle
\tag{\%{}t33}\label{t33} 
\mathit{d\ensuremath{\beta}}=\mathit{dx}\,\mathit{dy}\,\mathit{dz}\mbox{}
\]
%%%%%%%%%%%%%%%
\pagebreak


\textbf{Surface} $\vec{S}\in\mathbb{R}^3$



\noindent
%%%%%%%%%%%%%%%
%%% INPUT:
\begin{minipage}[t]{8ex}\color{red}\bf
(\%{}i34) 
\end{minipage}
\begin{minipage}[t]{\textwidth}\color{blue}\tt
ldisplay(S:3*[cos(u)*sin(v),sin(u)*sin(v),cos(v)])\$
\end{minipage}
%%% OUTPUT:
\[\displaystyle
\tag{\%{}t34}\label{t34} 
S=[3\cos{(u)}\,\sin{(v)},3\sin{(u)}\,\sin{(v)},3\cos{(v)}]\mbox{}
\]
%%%%%%%%%%%%%%%


\noindent
%%%%%%%%%%%%%%%
%%% INPUT:
\begin{minipage}[t]{8ex}\color{red}\bf
(\%{}i35) 
\end{minipage}
\begin{minipage}[t]{\textwidth}\color{blue}\tt
wxdraw3d(title="Surface",\\
         xu\_grid=100,yv\_grid=100,view=[60,30],\\
         proportional\_axes=xyz,surface\_hide=true,\\
         color=green,\\
         apply(parametric\_surface,append(S,[u,0,2*\ensuremath{\pi},v,0,\ensuremath{\frac{1}{2}}*\ensuremath{\pi}])))\$
\end{minipage}
%%% OUTPUT:
\[\displaystyle
\tag{\%{}t35}\label{t35} 
\includegraphics[width=.95\linewidth,height=.80\textheight,keepaspectratio]{Flux across a hemisphere_img/Flux across a hemisphere_2}\mbox{}
\]
%%%%%%%%%%%%%%%
\pagebreak



\noindent
%%%%%%%%%%%%%%%
%%% INPUT:
\begin{minipage}[t]{8ex}\color{red}\bf
(\%{}i36) 
\end{minipage}
\begin{minipage}[t]{\textwidth}\color{blue}\tt
ldisplay(S\_u:diff(S,u))\$
\end{minipage}
%%% OUTPUT:
\[\displaystyle
\tag{\%{}t36}\label{t36} 
{{S}_{u}}=[-3\sin{(u)}\,\sin{(v)},3\cos{(u)}\,\sin{(v)},0]\mbox{}
\]
%%%%%%%%%%%%%%%


\noindent
%%%%%%%%%%%%%%%
%%% INPUT:
\begin{minipage}[t]{8ex}\color{red}\bf
(\%{}i37) 
\end{minipage}
\begin{minipage}[t]{\textwidth}\color{blue}\tt
ldisplay(S\_v:diff(S,v))\$
\end{minipage}
%%% OUTPUT:
\[\displaystyle
\tag{\%{}t37}\label{t37} 
{{S}_{v}}=[3\cos{(u)}\,\cos{(v)},3\sin{(u)}\,\cos{(v)},-3\sin{(v)}]\mbox{}
\]
%%%%%%%%%%%%%%%

\textbf{Normal} $n_S\in\mathbb{R}^3$



\noindent
%%%%%%%%%%%%%%%
%%% INPUT:
\begin{minipage}[t]{8ex}\color{red}\bf
(\%{}i38) 
\end{minipage}
\begin{minipage}[t]{\textwidth}\color{blue}\tt
ldisplay(n\_S:trigsimp(mycross(S\_v,S\_u)))\$
\end{minipage}
%%% OUTPUT:
\[\displaystyle
\tag{\%{}t38}\label{t38} 
{{n}_{S}}=[9\cos{(u)}\,{{\sin{(v)}}^{2}},9\sin{(u)}\,{{\sin{(v)}}^{2}},9\cos{(v)}\,\sin{(v)}]\mbox{}
\]
%%%%%%%%%%%%%%%


\noindent
%%%%%%%%%%%%%%%
%%% INPUT:
\begin{minipage}[t]{8ex}\color{red}\bf
(\%{}i39) 
\end{minipage}
\begin{minipage}[t]{\textwidth}\color{blue}\tt
is(n\_S=3*sin(v)*S);
\end{minipage}
%%% OUTPUT:
\[\displaystyle
\tag{\%{}o39}\label{o39} 
\mbox{true}\mbox{}
\]
%%%%%%%%%%%%%%%

\textbf{Calculate} $\vec{F}\circ\vec{S}$



\noindent
%%%%%%%%%%%%%%%
%%% INPUT:
\begin{minipage}[t]{8ex}\color{red}\bf
(\%{}i40) 
\end{minipage}
\begin{minipage}[t]{\textwidth}\color{blue}\tt
ldisplay(FoS:subst(map("=",\ensuremath{\zeta},S),F))\$
\end{minipage}
%%% OUTPUT:
\[\displaystyle
\tag{\%{}t40}\label{t40} 
\mathit{FoS}=[3\sin{(u)}\,\sin{(v)},3\cos{(u)}\,\sin{(v)},3\cos{(v)}]\mbox{}
\]
%%%%%%%%%%%%%%%

\textbf{Calculate} $\alpha\circ\vec{S}$



\noindent
%%%%%%%%%%%%%%%
%%% INPUT:
\begin{minipage}[t]{8ex}\color{red}\bf
(\%{}i41) 
\end{minipage}
\begin{minipage}[t]{\textwidth}\color{blue}\tt
ldisplay(\ensuremath{\alpha}oS:subst(map("=",\ensuremath{\zeta},S),\ensuremath{\alpha}))\$
\end{minipage}
%%% OUTPUT:
\[\displaystyle
\tag{\%{}t41}\label{t41} 
\mathit{\ensuremath{\alpha}oS}=3\cos{(v)}\,\mathit{dz}+3\cos{(u)}\,\sin{(v)}\,\mathit{dy}+3\sin{(u)}\,\sin{(v)}\,\mathit{dx}\mbox{}
\]
%%%%%%%%%%%%%%%

\textbf{Integrand}



\noindent
%%%%%%%%%%%%%%%
%%% INPUT:
\begin{minipage}[t]{8ex}\color{red}\bf
(\%{}i42) 
\end{minipage}
\begin{minipage}[t]{\textwidth}\color{blue}\tt
integrand:trigsimp(n\_S|\ensuremath{\alpha}oS);
\end{minipage}
%%% OUTPUT:
\[\displaystyle
\tag{integrand}\label{integrand}
\left( 54\cos{(u)}\,\sin{(u)}-27\right) \,{{\sin{(v)}}^{3}}+27\sin{(v)}\mbox{}
\]
%%%%%%%%%%%%%%%


\noindent
%%%%%%%%%%%%%%%
%%% INPUT:
\begin{minipage}[t]{8ex}\color{red}\bf
(\%{}i43) 
\end{minipage}
\begin{minipage}[t]{\textwidth}\color{blue}\tt
integrand:trigsimp(FoS.n\_S);
\end{minipage}
%%% OUTPUT:
\[\displaystyle
\tag{integrand}\label{integrand}
\left( 54\cos{(u)}\,\sin{(u)}-27\right) \,{{\sin{(v)}}^{3}}+27\sin{(v)}\mbox{}
\]
%%%%%%%%%%%%%%%

\textbf{Flux integral}



\noindent
%%%%%%%%%%%%%%%
%%% INPUT:
\begin{minipage}[t]{8ex}\color{red}\bf
(\%{}i44) 
\end{minipage}
\begin{minipage}[t]{\textwidth}\color{blue}\tt
I:'integrate('integrate(integrand,v,0,\ensuremath{\frac{1}{2}}*\ensuremath{\pi}),u,0,2*\ensuremath{\pi})\$
\end{minipage}


\noindent
%%%%%%%%%%%%%%%
%%% INPUT:
\begin{minipage}[t]{8ex}\color{red}\bf
(\%{}i45) 
\end{minipage}
\begin{minipage}[t]{\textwidth}\color{blue}\tt
ldisplay(I=box(ev(I,integrate)))\$
\end{minipage}
%%% OUTPUT:
\[\displaystyle
\tag{\%{}t45}\label{t45} 
\int_{0}^{2\ensuremath{\pi} }{\left. \int_{0}^{\frac{\ensuremath{\pi} }{2}}{\left. \left( 54\cos{(u)}\,\sin{(u)}-27\right) \,{{\sin{(v)}}^{3}}+27\sin{(v)}dv\right.}du\right.}=\left( 18\ensuremath{\pi} \right) \mbox{}
\]
%%%%%%%%%%%%%%%
\pagebreak


\textbf{Using the Divergence theorem}
$$\iiint_E\nabla\cdot\vec{F}\,\mathrm{d}V=
\iint_{\partial E}\vec{F}\cdot\mathrm{d}\vec{S}=
\iint_S\vec{F}\cdot\mathrm{d}\vec{S}+
\iint_L\vec{F}\cdot\mathrm{d}\vec{S}$$


\textbf{Surface} $\vec{L}\in\mathbb{R}^3$



\noindent
%%%%%%%%%%%%%%%
%%% INPUT:
\begin{minipage}[t]{8ex}\color{red}\bf
(\%{}i46) 
\end{minipage}
\begin{minipage}[t]{\textwidth}\color{blue}\tt
ldisplay(L:[\ensuremath{\rho}*cos(\ensuremath{\theta}),\ensuremath{\rho}*sin(\ensuremath{\theta}),0])\$
\end{minipage}
%%% OUTPUT:
\[\displaystyle
\tag{\%{}t46}\label{t46} 
L=[\cos{\left( \mathit{\ensuremath{\theta}}\right) }\mathit{\ensuremath{\rho}},\sin{\left( \mathit{\ensuremath{\theta}}\right) }\mathit{\ensuremath{\rho}},0]\mbox{}
\]
%%%%%%%%%%%%%%%


\noindent
%%%%%%%%%%%%%%%
%%% INPUT:
\begin{minipage}[t]{8ex}\color{red}\bf
(\%{}i47) 
\end{minipage}
\begin{minipage}[t]{\textwidth}\color{blue}\tt
wxdraw3d(title="Surface",\\
         xu\_grid=100,yv\_grid=100,view=[60,30],\\
         proportional\_axes=xyz,surface\_hide=false,\\
         color=green,\\
         apply(parametric\_surface,append(S,[u,0,2*\ensuremath{\pi},v,0,\ensuremath{\frac{1}{2}}*\ensuremath{\pi}])),\\
         color=red,line\_width=5,\\
         apply(parametric\_surface,append(L,[\ensuremath{\rho},0,3,\ensuremath{\theta},0,2*\ensuremath{\pi}])))\$
\end{minipage}
%%% OUTPUT:
\[\displaystyle
\tag{\%{}t47}\label{t47} 
\includegraphics[width=.95\linewidth,height=.80\textheight,keepaspectratio]{Flux across a hemisphere_img/Flux across a hemisphere_3}\mbox{}
\]
%%%%%%%%%%%%%%%
\pagebreak



\noindent
%%%%%%%%%%%%%%%
%%% INPUT:
\begin{minipage}[t]{8ex}\color{red}\bf
(\%{}i48) 
\end{minipage}
\begin{minipage}[t]{\textwidth}\color{blue}\tt
ldisplay(L\_\ensuremath{\rho}:diff(L,\ensuremath{\rho}))\$
\end{minipage}
%%% OUTPUT:
\[\displaystyle
\tag{\%{}t48}\label{t48} 
{{L}_{\mathit{\ensuremath{\rho}}}}=[\cos{\left( \mathit{\ensuremath{\theta}}\right) },\sin{\left( \mathit{\ensuremath{\theta}}\right) },0]\mbox{}
\]
%%%%%%%%%%%%%%%


\noindent
%%%%%%%%%%%%%%%
%%% INPUT:
\begin{minipage}[t]{8ex}\color{red}\bf
(\%{}i49) 
\end{minipage}
\begin{minipage}[t]{\textwidth}\color{blue}\tt
ldisplay(L\_\ensuremath{\theta}:diff(L,\ensuremath{\theta}))\$
\end{minipage}
%%% OUTPUT:
\[\displaystyle
\tag{\%{}t49}\label{t49} 
{{L}_{\mathit{\ensuremath{\theta}}}}=[-\sin{\left( \mathit{\ensuremath{\theta}}\right) }\mathit{\ensuremath{\rho}},\cos{\left( \mathit{\ensuremath{\theta}}\right) }\mathit{\ensuremath{\rho}},0]\mbox{}
\]
%%%%%%%%%%%%%%%

\textbf{Normal} $n_L\in\mathbb{R}^3$



\noindent
%%%%%%%%%%%%%%%
%%% INPUT:
\begin{minipage}[t]{8ex}\color{red}\bf
(\%{}i50) 
\end{minipage}
\begin{minipage}[t]{\textwidth}\color{blue}\tt
ldisplay(n\_L:trigsimp(mycross(L\_\ensuremath{\theta},L\_\ensuremath{\rho})))\$
\end{minipage}
%%% OUTPUT:
\[\displaystyle
\tag{\%{}t50}\label{t50} 
{{n}_{L}}=[0,0,-\mathit{\ensuremath{\rho}}]\mbox{}
\]
%%%%%%%%%%%%%%%

\textbf{Calculate} $\vec{F}\circ\vec{L}$



\noindent
%%%%%%%%%%%%%%%
%%% INPUT:
\begin{minipage}[t]{8ex}\color{red}\bf
(\%{}i51) 
\end{minipage}
\begin{minipage}[t]{\textwidth}\color{blue}\tt
ldisplay(FoL:subst(map("=",\ensuremath{\zeta},L),F))\$
\end{minipage}
%%% OUTPUT:
\[\displaystyle
\tag{\%{}t51}\label{t51} 
\mathit{FoL}=[\sin{\left( \mathit{\ensuremath{\theta}}\right) }\mathit{\ensuremath{\rho}},\cos{\left( \mathit{\ensuremath{\theta}}\right) }\mathit{\ensuremath{\rho}},0]\mbox{}
\]
%%%%%%%%%%%%%%%

\textbf{Calculate} $\alpha\circ\vec{L}$



\noindent
%%%%%%%%%%%%%%%
%%% INPUT:
\begin{minipage}[t]{8ex}\color{red}\bf
(\%{}i52) 
\end{minipage}
\begin{minipage}[t]{\textwidth}\color{blue}\tt
ldisplay(\ensuremath{\alpha}oL:subst(map("=",\ensuremath{\zeta},L),\ensuremath{\alpha}))\$
\end{minipage}
%%% OUTPUT:
\[\displaystyle
\tag{\%{}t52}\label{t52} 
\mathit{\ensuremath{\alpha}oL}=\mathit{dx}\,\sin{\left( \mathit{\ensuremath{\theta}}\right) }\mathit{\ensuremath{\rho}}+\mathit{dy}\,\cos{\left( \mathit{\ensuremath{\theta}}\right) }\mathit{\ensuremath{\rho}}\mbox{}
\]
%%%%%%%%%%%%%%%

\textbf{Integrand}



\noindent
%%%%%%%%%%%%%%%
%%% INPUT:
\begin{minipage}[t]{8ex}\color{red}\bf
(\%{}i53) 
\end{minipage}
\begin{minipage}[t]{\textwidth}\color{blue}\tt
integrand:trigsimp(n\_L|\ensuremath{\alpha}oL);
\end{minipage}
%%% OUTPUT:
\[\displaystyle
\tag{integrand}\label{integrand}
0\mbox{}
\]
%%%%%%%%%%%%%%%


\noindent
%%%%%%%%%%%%%%%
%%% INPUT:
\begin{minipage}[t]{8ex}\color{red}\bf
(\%{}i54) 
\end{minipage}
\begin{minipage}[t]{\textwidth}\color{blue}\tt
integrand:trigsimp(FoL.n\_L);
\end{minipage}
%%% OUTPUT:
\[\displaystyle
\tag{integrand}\label{integrand}
0\mbox{}
\]
%%%%%%%%%%%%%%%
\pagebreak


\textbf{Spherical coordinates}



\noindent
%%%%%%%%%%%%%%%
%%% INPUT:
\begin{minipage}[t]{8ex}\color{red}\bf
(\%{}i58) 
\end{minipage}
\begin{minipage}[t]{\textwidth}\color{blue}\tt
assume(0\ensuremath{\leq}\ensuremath{\rho})\$\\
assume(0\ensuremath{\leq}\ensuremath{\theta},\ensuremath{\theta}\ensuremath{\leq}\ensuremath{\pi})\$\\
assume(0\ensuremath{\leq}sin(\ensuremath{\theta}))\$\\
assume(0\ensuremath{\leq}\ensuremath{\phi},\ensuremath{\phi}\ensuremath{\leq}2*\ensuremath{\pi})\$
\end{minipage}


\noindent
%%%%%%%%%%%%%%%
%%% INPUT:
\begin{minipage}[t]{8ex}\color{red}\bf
(\%{}i59) 
\end{minipage}
\begin{minipage}[t]{\textwidth}\color{blue}\tt
\ensuremath{\xi}:[\ensuremath{\rho},\ensuremath{\theta},\ensuremath{\phi}]\$
\end{minipage}


\noindent
%%%%%%%%%%%%%%%
%%% INPUT:
\begin{minipage}[t]{8ex}\color{red}\bf
(\%{}i60) 
\end{minipage}
\begin{minipage}[t]{\textwidth}\color{blue}\tt
ldisplay(E:[\ensuremath{\rho}*sin(\ensuremath{\theta})*cos(\ensuremath{\phi}),\ensuremath{\rho}*sin(\ensuremath{\theta})*sin(\ensuremath{\phi}),\ensuremath{\rho}*cos(\ensuremath{\theta})])\$
\end{minipage}
%%% OUTPUT:
\[\displaystyle
\tag{\%{}t60}\label{t60} 
E=[\sin{\left( \mathit{\ensuremath{\theta}}\right) }\mathit{\ensuremath{\rho}}\,\cos{\left( \mathit{\ensuremath{\phi}}\right) },\sin{\left( \mathit{\ensuremath{\theta}}\right) }\mathit{\ensuremath{\rho}}\,\sin{\left( \mathit{\ensuremath{\phi}}\right) },\cos{\left( \mathit{\ensuremath{\theta}}\right) }\mathit{\ensuremath{\rho}}]\mbox{}
\]
%%%%%%%%%%%%%%%


\noindent
%%%%%%%%%%%%%%%
%%% INPUT:
\begin{minipage}[t]{8ex}\color{red}\bf
(\%{}i61) 
\end{minipage}
\begin{minipage}[t]{\textwidth}\color{blue}\tt
scalefactors(append([E],\ensuremath{\xi}))\$
\end{minipage}


\noindent
%%%%%%%%%%%%%%%
%%% INPUT:
\begin{minipage}[t]{8ex}\color{red}\bf
(\%{}i62) 
\end{minipage}
\begin{minipage}[t]{\textwidth}\color{blue}\tt
sf;
\end{minipage}
%%% OUTPUT:
\[\displaystyle
\tag{\%{}o62}\label{o62} 
[1,\mathit{\ensuremath{\rho}},\sin{\left( \mathit{\ensuremath{\theta}}\right) }\mathit{\ensuremath{\rho}}]\mbox{}
\]
%%%%%%%%%%%%%%%


\noindent
%%%%%%%%%%%%%%%
%%% INPUT:
\begin{minipage}[t]{8ex}\color{red}\bf
(\%{}i63) 
\end{minipage}
\begin{minipage}[t]{\textwidth}\color{blue}\tt
sfprod;
\end{minipage}
%%% OUTPUT:
\[\displaystyle
\tag{\%{}o63}\label{o63} 
\sin{\left( \mathit{\ensuremath{\theta}}\right) }\,{{\mathit{\ensuremath{\rho}}}^{2}}\mbox{}
\]
%%%%%%%%%%%%%%%


\noindent
%%%%%%%%%%%%%%%
%%% INPUT:
\begin{minipage}[t]{8ex}\color{red}\bf
(\%{}i64) 
\end{minipage}
\begin{minipage}[t]{\textwidth}\color{blue}\tt
dimension;
\end{minipage}
%%% OUTPUT:
\[\displaystyle
\tag{\%{}o64}\label{o64} 
3\mbox{}
\]
%%%%%%%%%%%%%%%


\noindent
%%%%%%%%%%%%%%%
%%% INPUT:
\begin{minipage}[t]{8ex}\color{red}\bf
(\%{}i65) 
\end{minipage}
\begin{minipage}[t]{\textwidth}\color{blue}\tt
ldisplay(J:trigsimp(determinant(jacobian(E,\ensuremath{\xi}))))\$
\end{minipage}
%%% OUTPUT:
\[\displaystyle
\tag{\%{}t65}\label{t65} 
J=\sin{\left( \mathit{\ensuremath{\theta}}\right) }\,{{\mathit{\ensuremath{\rho}}}^{2}}\mbox{}
\]
%%%%%%%%%%%%%%%


\noindent
%%%%%%%%%%%%%%%
%%% INPUT:
\begin{minipage}[t]{8ex}\color{red}\bf
(\%{}i66) 
\end{minipage}
\begin{minipage}[t]{\textwidth}\color{blue}\tt
trigsimp(diff(E,\ensuremath{\phi})|(diff(E,\ensuremath{\theta})|(diff(E,\ensuremath{\rho})|d\ensuremath{\beta})));
\end{minipage}
%%% OUTPUT:
\[\displaystyle
\tag{\%{}o66}\label{o66} 
\sin{\left( \mathit{\ensuremath{\theta}}\right) }\,{{\mathit{\ensuremath{\rho}}}^{2}}\mbox{}
\]
%%%%%%%%%%%%%%%

\textbf{Calculate} $$\iiint_E\nabla\cdot\vec{F}\,\mathrm{d}V$$



\noindent
%%%%%%%%%%%%%%%
%%% INPUT:
\begin{minipage}[t]{8ex}\color{red}\bf
(\%{}i67) 
\end{minipage}
\begin{minipage}[t]{\textwidth}\color{blue}\tt
I:\ensuremath{\frac{1}{2}}*'integrate('integrate('integrate(divF*J,\ensuremath{\rho},0,3),\ensuremath{\theta},0,\ensuremath{\pi}),\ensuremath{\phi},0,2*\ensuremath{\pi})\$
\end{minipage}


\noindent
%%%%%%%%%%%%%%%
%%% INPUT:
\begin{minipage}[t]{8ex}\color{red}\bf
(\%{}i68) 
\end{minipage}
\begin{minipage}[t]{\textwidth}\color{blue}\tt
ldisplay(I=box(ev(I,integrate)))\$
\end{minipage}
%%% OUTPUT:
\[\displaystyle
\tag{\%{}t68}\label{t68} 
\ensuremath{\pi} \int_{0}^{\ensuremath{\pi} }{\left. \sin{\left( \mathit{\ensuremath{\theta}}\right) }d\mathit{\ensuremath{\theta}}\right.}\,\int_{0}^{3}{\left. {{\mathit{\ensuremath{\rho}}}^{2}}d\mathit{\ensuremath{\rho}}\right.}=\left( 18\ensuremath{\pi} \right) \mbox{}
\]
%%%%%%%%%%%%%%%
\end{document}
